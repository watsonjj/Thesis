\chapter{\label{ch8-onsky}On-Sky Observations} 

\minitoc

\notes[inline,caption={}]{
	\section{Plan}
	\subsection{Topics}
	\begin{itemize}
		\item Decided upon reduction methods
		\item Potentially different than for performance chapter
		\item CHEC-M campaign
		\item MC CHEC-S
		\item Future observations
		\item Jupiter observations (Beyond cherenkov?)
	\end{itemize}
	\subsection{Questions}
	\begin{itemize}
		\item ?
	\end{itemize}
}

\section{Introduction}

\begin{figure}
  \includegraphics[width=\textwidth]{akira-telescope}
  \caption[Photo of CHEC-M installed on the GCT telescope structure.]{Photo of CHEC-M installed on the GCT telescope structure, taken during the first on-telescope campaign \cite{akira-telescope}.}
  \label{fig:akira-telescope}
\end{figure}

\begin{figure}
  \includegraphics[width=\textwidth]{akira-reflection}
  \caption[Photo of the reflection of CHEC-M in the secondary mirror.]{Photo of the reflection of CHEC-M in the GCT telescope's secondary mirror, taken during the first on-telescope campaign \cite{akira-reflection}.}
  \label{fig:akira-reflection}
\end{figure}

Testing the operation of the camera on the telescope structure is a very important part of the commissioning procedure. When on-telescope, the camera is in an environment we have very little control over. It is exposed to factors such as weather conditions and excessive \gls{nsb} from various sources (including moonlight, starlight, and artificial light pollution). The on-telescope campaigns are therefore a useful measure of the robustness of the camera, and the procedures used to operate the the telescope as a whole.

The first on-telescope campaign took place during November 2015, at the location of the \gls{gct} telescope structure (Observatoire de Paris-Meudon), just before the inauguration of the \gls{gct} prototype. The primary intention of this campaign was to test the integration and operation procedure for \gls{chec-m} on the telescope structure, however the first detection of Cherenkov light from atmospheric showers by a \gls{cta} prototype camera was also achieved \cite{Watson2017}. 

After returning to the lab for further testing and characterisation, \gls{chec-m} was then re-installed on the \gls{gct} structure in March 2017 for a second on-telescope campaign. During this second campaign the \gls{gct} telescope was pointed towards two \gls{vhe} gamma-ray sources, Mrk421 and Mrk501. These two sources are blazar objects (\gls{agn} with a relativistic jet directed towards Earth) and were the first extragalactic \si{TeV} sources to be discovered \cite{Punch1992, Quinn1996}, testifying to their brightness. However, due to the high \gls{nsb} background that is present at the Meudon site (20 to 100 times brighter than expected at the final \gls{cta} site), the camera had to be operated at a low gain and high trigger threshold \cite{Zorn2017}. The former setting was used as a precaution to avoid damage to the \glspl{mapmt}, and the latter to avoid triggering on the \gls{nsb} photons. The combination of these operating conditions, and the limited observation time, meant an astrophysical detection was unlikely for this campaign. Nevertheless, Cherenkov showers were detected during the campaign. This chapter will describe the results I have obtained from the images of these Cherenkov showers.

\section{Cherenkov Shower Images}

Utilising the calibration procedure defined in Chapter~\ref{ch5-calibration}, and a simple integration window combined with the \textit{Neighbour Peak Finding} technique described in Chapter~\ref{ch6-reduction}, the Cherenkov signal in each pixel was extracted for every trigger event. The results shown in this section originate from a 30-minute-long observation of Mrk501, during which the camera received triggers at a rate of \SI{\sim 0.1}{Hz}.

\begin{figure}
  \includegraphics[width=\textwidth]{hillas_checm_width_length}
  \caption[\textit{Hillas} length versus width for an on-sky observing with CHEC-M.]{The \textit{Hillas} length versus width of each image in a 30-minute-long observation of Mrk501 with CHEC-M on the \gls{gct} telescope structure (labelled as ``Real''). Simulations of CHEC-M on the \gls{gct} telescope are also included for comparison. Four different types of events have been highlighted and assigned a category based on a manual examination of their camera image.}
  \label{fig:hillas_checm_width_length}
\end{figure}

\begin{sidewaysfigure}
  \begin{subfigure}[b]{0.49\textwidth}
    \includegraphics[width=\textwidth]{hillas_checm_typical}
    \caption{Typical Shower}
    \label{fig:hillas_checm_typical}
  \end{subfigure}
  \hfill
  \begin{subfigure}[b]{0.49\textwidth}
    \includegraphics[width=\textwidth]{hillas_checm_bright}
    \caption{Bright Shower}
    \label{fig:hillas_checm_bright}
  \end{subfigure}
   \hfill
  \begin{subfigure}[b]{0.49\textwidth}
    \includegraphics[width=\textwidth]{hillas_checm_cr}
    \caption{Direct CR}
    \label{fig:hillas_checm_cr}
  \end{subfigure}
  \hfill
  \begin{subfigure}[b]{0.49\textwidth}
    \includegraphics[width=\textwidth]{hillas_checm_cr_ee}
    \caption{Direct CR Entry\&Exit}
    \label{fig:hillas_checm_cr_ee}
  \end{subfigure}
  \caption[Selection of Cherenkov images.]{A selection of images taken by \gls{chec-m} during its second on-telescope campaign. The images chosen correspond to the individually highlighted events in Figure~\ref{fig:hillas_checm_width_length}.}
\end{sidewaysfigure}

Figure~\ref{fig:hillas_checm_width_length} displays the distribution of \textit{Hillas} width and length for the detected (Section~\ref{section:image_parametrisation}). Four different 

out of focus



\section{Jupiter Observations}

\section{Future}

