\chapter{\label{ch7-performance}Camera Performance} 

\minitoc

\section{Introduction}

As discussed in Chapter~\ref{ch3-architecture}, it is important that the \gls{chec} camera meets certain criteria in order for it to be accepted as an in-kind contribution to \gls{cta}. One important subset of the criteria is the camera's performance. The requirements that must be fulfilled are driven by the science goal of \gls{cta}. If a camera does not meet the requirements laid out by the \gls{cta} observatory, then it will be refused as a contribution in its current state, lest the science goals of \gls{cta} are not achieved.

This chapter will cover many of the primary standards used to assess a \gls{cta} camera's performance. The results shown in this chapter are all my own, and are obtained using the procedures defined in the preceding chapters. Two important aspects of the camera's performance which are not included in this chapter are the trigger efficiency and absolute photon detection efficiency. The investigations into these parameters are ongoing, of which I have not had direct involvement with as part of my DPhil.

\section{CHEC-S Monte Carlo Model}

An important preface to reporting on the performance of \gls{chec} is the description of the efforts in generating an accurate model of the camera for use inside the Monte Carlo simulations performed by \pkg{sim\_telarray} (Chapter~\ref{ch4-software}). These simulations allow us to more widely explore the parameter space related to our camera performance, and identify potential issues that cause the real camera to drift away from ideal operation. It is also a necessary step in investigating the on-sky performance of the telescope, as:

\begin{itemize}
\item One does not have prior knowledge of the properties of the light incident on the pixels in the real world.
\item The camera is still being tested in the lab. An on-sky campaign for \gls{chec-s} on the \gls{astri} telescope structure is planned for later this year.
\end{itemize}

Contained within the lab data are the parameters required for the Monte Carlo validation process. Important parameters include:

\begin{itemize}
\item Pixel position on the focal surface,
\item Pulse shape for a single photoelectron,
\item Trigger discrimination behaviour,
\item Quantum efficiency (or \gls{pde}),
\item Variation of quantum efficiency between pixels
\item Electronic baseline variation,
\item Photosensor gain,
\item Variation of gain between pixels,
\item \gls{spe} spectrum shape.
\end{itemize}

\begin{figure}
	\centering
    \includegraphics[width=\textwidth]{spe_sim_lab} 
	\caption[Comparison of the SPE spectra between lab measurements and simulations.]{Comparison of the SPE spectra for a single pixel between lab measurements and simulations after an initial attempt towards a more accurate Monte Carlo model. The SPE spectra are identically extracted in both cases, using the \textit{Cross Correlation} charge extraction method. Both spectra are normalised in the x direction by their respective single-photoelectron value, and in the y direction such that their integral is 1.}
	\label{fig:spe_sim_lab}
\end{figure}

\begin{table}[h!]
\centering
\begin{tabular}{ll|ll} \toprule
    Fit Parameter        &            & Simulation          & Lab                \\ \midrule
    Average Illumination & [\si{\pe}] & 0.717 $\pm$ 0.011  & 0.751 $\pm$ 0.011 \\
    Pedestal Deviation   & [\si{\pe}] & 0.314 $\pm$ 0.003  & 0.286 $\pm$ 0.002 \\
    Gain Deviation       & [\si{\pe}] & 0.109 $\pm$ 0.009  & 0.078 $\pm$ 0.007 \\
    Optical Crosstalk    &            & 0.387 $\pm$ 0.007  & 0.350 $\pm$ 0.006 \\ \bottomrule
\end{tabular}
\caption{Parameter values resulting from the fit to the spectra in Figure~\ref{fig:spe_sim_lab}. The \si{1}{$\sigma$} parabolic errors obtained from the covariance matrix of the fit parameters are quoted.}
\label{table:spe_sim_lab}
\end{table}

For the simulation results presented in this thesis, an updated value was obtained for as many of the relevant \pkg{sim\_telarray} parameters as possible. Figure~\ref{fig:spe_sim_lab} displays the resulting differences in terms of the \gls{spe} spectra between lab and simulation. The discrepancies in parameters resulting from the fits to the \gls{spe} spectra are shown in Table~\ref{table:spe_sim_lab}, and are deemed to be close enough for the investigations in this thesis. Further details about the fit procedure for \gls{spe} spectra can be found in Appendix~\ref{a1-spe}. The differences between lab and simulation in other factors and at higher amplitudes are explored through \textit{Charge Resolution} comparisons, investigated later in this chapter.

\section{CHEC-S Charge Resolution}

The \textit{Charge Resolution} is the principle criterion used within \gls{cta} to express how well the camera can resolve a signal. The concept of \textit{Charge Resolution} is introduced in Section~\ref{section:cr}, alongside the \gls{cta} requirement \requirementref{B-TEL-1010 Charge Resolution}. It not only measures the quality of the camera's photosensor and electronics, but also the aptitude of the calibration and signal extraction. Consequently, obtaining a \textit{Charge Resolution} of the camera that meets this requirement has been the underlying driver behind my efforts in developing the techniques described in this thesis.

\subsection{Procedure and Datasets}

As directed in the \requirementref{B-TEL-1010 Charge Resolution} requirement, one must validate \textit{Charge Resolution} results in three ways. To achieve this, and understand the relationship between the three validation approaches, four procedures are used to obtain a \textit{Charge Resolution}. The procedures, and their associated name for the purpose of this thesis, are:

\begin{description}
\item [Lab] Utilises uniform illumination datasets taken in the lab that cover the full dynamic range of the camera, with 1000 events at each illumination. The average expected charge per pixel (in photoelectrons) for each illumination is calibrated by the procedure detailed in Section~\ref{section:lab-calib}. Using the calibration procedures detailed in Chapter~\ref{ch5-calibration} and the \textit{Cross Correlation} charge extraction technique (Chapter~\ref{ch6-reduction}), a value of measured charge in photoelectrons is obtained for every waveform. As the average expected charge values include the Poisson fluctuations, it is appropriate to use Equation~\ref{eq:charge_res} for calculating the \textit{Charge Resolution}, with the measured charge per waveform used for ${Q_M}_i$ and the average expected charge for $Q_T$.
\item [MCLab] Simulations of the dynamic range datasets from the lab are obtained with the updated \pkg{sim\_telarray} camera model. Using the identical method used for the lab data (excluding the \gls{target} calibration) the charge is extracted and calibrated from the waveforms. The average expected charge for each illumination per pixel is also obtained in the same way as the previous procedure. However, the simulations contain perfect laser uniformity across the camera face (the geometry shown in Figure~\ref{fig:laser_geometry} is still applicable). This dataset then fully represents the same measurements, but with the Monte Carlo model of the camera instead of the physical camera. With an accurate model of the camera, the \textit{Charge Resolution} result should be the same as from the lab measurements. Equation~\ref{eq:charge_res} is also used in this procedure for calculating the \textit{Charge Resolution}.
\item [MCLabTrue] Using the same dataset as the previous procedure, but instead of using the average expected charge, the true number of photoelectrons that were incident on the pixel for each waveform are extracted from the simulation file. The linear fit between the average measured charge and the ``true charge'' is used to calibrate the extracted charge into corresponding units. As the unique value of ``true charge'' per waveform is now used as $Q_T$, Equation~\ref{eq:charge_res2} must be used to account for the lack of Poisson fluctuations in $Q_T$. The \textit{Charge Resolution} resulting from this procedure demonstrates the change in going from ``average expected charge'' to ``true charge'', which is important for interpreting the results from the next procedure.
\item [MCOnsky] A dataset containing Monte Carlo simulations of air showers, detected by \gls{chec} on the \gls{astri} telescope structure is made using \pkg{CORSIKA} and \pkg{sim\_telarray}. The \textit{Charge Resolution} is then calculated with the same procedure as the previous one. This procedure is the definitive approach for assessment of \textit{Charge Resolution}, as defined in the requirements. The previous procedures exist to give support to this result.
\end{description}

A single pixel at the centre of the camera, 888, is used for this investigation. For some of the initial \textit{Charge Resolution} results, the spread in \textit{Charge Resolution} across the other pixels in the camera (excluding ``dead'' pixels) is indicated through the error bars in the y axis.

\subsection{Lab Results}

\begin{figure}
	\centering
    \includegraphics[width=\textwidth]{checs_mve_hist} 
	\caption[]{}
	\label{fig:checs_mve_hist}
\end{figure}

\begin{figure}
	\centering
    \includegraphics[width=\textwidth]{checs_mve_scatter} 
	\caption[]{}
	\label{fig:checs_mve_scatter}
\end{figure}

\begin{figure}
	\centering
    \includegraphics[width=\textwidth]{cr_1_lab_raw} 
	\caption[]{}
	\label{fig:cr_1_lab_raw}
\end{figure}

\begin{figure}
	\centering
    \includegraphics[width=\textwidth]{cr_1_lab} 
	\caption[]{}
	\label{fig:cr_1_lab}
\end{figure}

\notes[inline,caption={}]{
	\begin{itemize}
		\item Measured versus True
        \item Average versus True
        \item Charge resolution
        \item Charge resolution / Requirement
        \item Justification of the lab chargeres using equation with ENF
	\end{itemize}
}

\subsection{Lab versus Monte Carlo}

\notes[inline,caption={}]{
	\begin{itemize}
		\item Differences between Lab and MC
        \begin{itemize}
			\item Saturation
            \item TFs
            \item Timing
            \item Electrical Crosstalk
            \item OPCT
            \item Reference Pulse shape
            \begin{itemize}
            	\item Can see change in FWHM
        		\item CE investigation excludes this contribution though
			\end{itemize}
		\end{itemize}
        \item 
	\end{itemize}
}
\notes[inline]{Remember to talk about how the MC charge resolution provides us insight into performance with perfect Transfer Functions}
\notes[inline]{big diff between MC and the lab is the potential cross talk in the electronics and the effect e.g. of ground bounce - difference in full camera illumination}

\subsection{Night Sky Background}

\notes[inline,caption={}]{
	\begin{itemize}
		\item MCLab
        \item MCOnsky
	\end{itemize}
}

\subsection{Optical Crosstalk}

\notes[inline,caption={}]{
	\begin{itemize}
		\item MC
        \item Changing bias voltage - meets requirements, but at the cost of cherenkov shower resolution, reference ch3. Need to reduce opct without reducing bias voltage - better detectors
	\end{itemize}
}

\subsection{Conclusion}

\notes[inline,caption={}]{
	\begin{itemize}
		\item Can we meet requirement with lower opct? show with enf equation at 125 MHz
	\end{itemize}
}

\section{CHEC-S Pulse Shape}

\section{CHEC-S Time Resolution}

\section{CHEC-M}

\notes[inline]{Comparison in performance between CHEC-M and CHEC-S. show SPE spectrum again, huge decrease in spe\_sigma}



