\chapter{\label{ch7-performance}Camera Performance} 

\minitoc

\notes[inline,caption={}]{
	\section{Plan}
	\subsection{Topics}
	\begin{itemize}
		\item Charge Resolution
		\item TF Investigations
		\item Different NSB
		\item MC Validation
		\item MC Performance
	\end{itemize}
	\subsection{Questions}
	\begin{itemize}
		\item What other criteria?
		\begin{itemize}
			\item Trigger performance - even though I haven't contributed
		\end{itemize}
	\end{itemize}
}

\section{Introduction}

\section{Pulse Shape}

\section{Timing Characteristics}

\section{MC Validation}

\change[inline]{big diff between MC and the lab is the potential cross talk in the electronics and the effect e.g. of ground bounce - difference in full camera illumination}


\section{Charge Resolution}

The standard low-level criterion for performance used in \gls{cta} is the \textit{Charge Resolution}. It encompasses both the bias and the standard deviation of the extracted charge versus the expected charge to provide a measure of the waveform, calibration, and charge reconstruction quality. The \gls{cta} requirement \change[inline]{requirement label} (Chapter~\ref{ch3-architecture}) defines a limit which must be met for resolving the signal for any telescope in \gls{cta}.



\section{Conclusion}