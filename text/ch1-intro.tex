\chapter{\label{ch1-intro}Introduction} 

\minitoc

\notes[inline,caption={}]{
	\section{Plan}
	\subsection{Topics}
	\begin{itemize}
		\item High Energy Astrophysics
		\begin{itemize}
			\item Fermi
			\item Fermi Bubbles
			\item HAWC
		\end{itemize}
		\item IACTs
		\item CTA
		\item CTA Science
		\begin{itemize}
			\item Science Cases
			\item Use "Science with CTA" paper
		\end{itemize}
		\item SSTs
		\item SST Science
		\begin{itemize}
			\item What do we contribute?
			\item What can't be done without us?
		\end{itemize}
		\item GCT
		\item CHEC
		\begin{itemize}
			\item What makes us better?
			\item Advantages of Schwarzchild-Couder
			\begin{itemize}
				\item Increased FoV
				\item Size
				\item Cost
			\end{itemize}
			\item Advantages of full waveform readout
			\item Other Advantages?
			\begin{itemize}
				\item Trigger
				\item Energy/power/voltage Requirements
				\item Commonalities (SCT)
			\end{itemize}
		\end{itemize}
	\end{itemize}
	\subsection{Questions}
	\begin{itemize}
		\item ?
	\end{itemize}
}

\change[inline]{Shower properties, photons from bottom of shower are received before those at the top as the particle travels faster than light. Good figure in \cite{Cogan2006}}
\change[inline]{Gamma/Hadron/Lepton}

\change[inline]{Terminology note: charge not used in terms of columns, it refers to counts of photoelectrons, for which mV and ADC are a proxy of. Do a ctrl-f at end to check how charge is used}

\section{Cherenkov Shower Characteristics}

\subsection{Photon Arrival Time} \label{section:photon_arrival_time}

\change[inline]{Quoted from Holder2005: The longitudinal development of an air shower is reflected in the long axis of the elliptical image recorded in the camera. The photon arrival time profile along this axis is largely a result of geometrical path length differences, and hence the shower core distance. As the shower particles move faster than the speed of light in air, when the shower has a small core distance Cherenkov light emitted from lower in the atmosphere is received at the telescope first. At large core distances, this situation is reversed, as the Cherenkov light travel time from the shower to the telescope dominates. The effect of this is to produce a timing gradient along the long axis of the image, the size and sign of which depend upon the core distance. For gamma-ray showers from a point source at the centre of the field of view, the shower core distance is directly related to the angular distance in the camera of the image from the source position.
Figure}