The interaction of very high energy gamma rays with the Earth's atmosphere produces extensive electromagnetic particle cascades. This superluminal shower of particles emits photons of blue wavelength, commonly known as Cherenkov radiation. The \gls{iact} technique enables the probing of the universe at \si{TeV} energies through the detection of these Cherenkov showers. The \gls{cta} will represent the next leap forward in \gls{vhe} astronomy, improving on the sensitivity of current \glspl{iact} by a factor of 10, encompassing energies from \SI{20}{GeV} to \SI{300}{TeV}, and operating as the first open observatory in this field. A major component of \gls{cta} are the \glspl{sst}, a necessary ingredient in exploring past the present frontier in gamma-ray astronomy.

One of three proposed designs for the \gls{sst} is the \gls{gct}. Utilising a dual-mirror Schwarzschild-Couder optical design, \gls{gct} enables a \SI{9}{\degree} \gls{fov} for a compact camera design. The camera developed for \gls{gct} is the \gls{chec}. Two prototypes for \gls{chec} have been built, each utilising different compact photosensor technology. \gls{chec-m} features \glspl{mapmt}, a pixelised extension of the \gls{pmt} technology extensively used by \glspl{iact}. \gls{chec-s} features \glspl{sipmt}, a novel photosensor which utilises semiconductor technology for high resolution photon counting over a large dynamic range. To fully utilise the signal output from these photosensors, and allow the opportunity for future data analysis procedures to be exploited, the entire signal received from these photosensors are digitised into waveforms following a trigger. These waveforms have a length of 96 samples with nanosecond precision.

In this thesis, the full calibration and signal-extraction pipeline currently adopted by \gls{chec} to reliably extract the Cherenkov signal from the waveforms is presented. The resulting performance of these procedures, and of the camera designs, is explored with respect to the requirements specified by the \gls{cta} Observatory. Potential improvements to the camera and calibration implementations are identified, and simulations of \gls{chec} are utilised to demonstrate the performance increase these proposals provide. Consequently, an improvement to the photosensor that would allow the \gls{chec-s} prototype design to comfortably meet the \gls{cta} requirements is specified. Testing of these improvements is anticipated to commence in early 2019. Finally, the results of the second on-telescope campaign for the \gls{chec-m} prototype are presented, during which observations of Cherenkov showers and optical measurements of Jupiter were conducted.