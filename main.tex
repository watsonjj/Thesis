%%%%%%%%%%%%%%%%%%%%%%%%%%%%%%%%%%%%%%%%%%%%%%%%%%%%%%%%%%%%%%%
%% OXFORD THESIS TEMPLATE

% Use this template to produce a standard thesis that meets the Oxford University requirements for DPhil submission
%
% Originally by Keith A. Gillow (gillow@maths.ox.ac.uk), 1997
% Modified by Sam Evans (sam@samuelevansresearch.org), 2007
% Modified by John McManigle (john@oxfordechoes.com), 2015
% Modified by Jason Watson (jason.watson@physics.ox.ac.uk), 2018
%
% This version Copyright (c) 2015-2017 John McManigle
%
% Broad permissions are granted to use, modify, and distribute this software
% as specified in the MIT License included in this distribution's LICENSE file.
%

%%%%% PAGE LAYOUT
% This one will format for two-sided binding (ie left and right pages have mirror margins; blank pages inserted where needed):
%\documentclass[a4paper,twoside]{ociamthesis}
% This one will format for one-sided binding (ie left margin > right margin; no extra blank pages):
%\documentclass[a4paper]{ociamthesis}
% This one will format for PDF output (ie equal margins, no extra blank pages):
\documentclass[a4paper,nobind]{ociamthesis} 


%%%%% PGF
\usepackage{pgf}
%\pgfplotsset{compat=1.12}
\DeclareUnicodeCharacter{2212}{-}


%%%%% DRAFT FOOTER
\fancyfoot[C]{\emph{DRAFT Printed on \today}}  


%%%%% DRAFT CORRECTIONS
% This highlights (in blue) corrections marked with (for words) \mccorrect{blah} or (for whole
% paragraphs) \begin{mccorrection} . . . \end{mccorrection}.  This can be useful for sending a PDF of
% your corrected thesis to your examiners for review.  Turn it off, and the blue disappears.
\correctionstrue


%%%%% BIBLIOGRAPHY SETUP
% Note that your bibliography will require some tweaking depending on your department, preferred format, etc.
% The options included below are just very basic "sciencey" and "humanitiesey" options to get started.
% If you've not used LaTeX before, I recommend reading a little about biblatex/biber and getting started with it.
% If you're already a LaTeX pro and are used to natbib or something, modify as necessary.
% Either way, you'll have to choose and configure an appropriate bibliography format...

% The science-type option: numerical in-text citation with references in order of appearance.
\usepackage[style=numeric-comp, sorting=none, backend=biber, doi=false, isbn=false]{biblatex}
\newcommand*{\bibtitle}{References}

% This makes the bibliography left-aligned (not 'justified') and slightly smaller font.
\renewcommand*{\bibfont}{\raggedright\small}

% Change this to the name of your .bib file (usually exported from a citation manager like Zotero or EndNote).
\addbibresource{bibtex/articles.bib}
\addbibresource{bibtex/internal.bib}
\addbibresource{bibtex/webpages.bib}


% Uncomment this if you want equation numbers per section (2.3.12), instead of per chapter (2.18):
\numberwithin{equation}{subsection}


%%%%% THESIS / TITLE PAGE INFORMATION
\title{Calibration and Analysis of the GCT Camera for the Cherenkov Telescope Array}
\author{Jason J. Watson}
\college{Brasenose College}
\degree{Doctor of Philosophy}
\degreedate{Trinity 2018}


%%%%% PERSONAL MACROS
% Commands
\newcommand{\quotes}[1]{``#1''}

% New list method
\newlength{\bulletwidth}\settowidth{\bulletwidth}{$\bullet$}
\newcommand{\mitem}{\setlength{\leftskip}{\leftmargin}\hspace*{-\labelsep}\hspace*{-\bulletwidth}$\bullet$\hspace*{\labelsep}}
\newcommand{\mend}{\setlength{\leftskip}{0cm}}
\newcommand{\utilde}{\raise.17ex\hbox{$\scriptstyle\mathtt{\sim}$}}

% Abbreviations:s
\usepackage[toc,acronym,nomain,shortcuts,nopostdot]{glossaries-prefix}
%\newcommand*\glsr[2][]{\glsdisp[#1]{#2}{\glsentryshort{#2} (\glsentrylong{#2})}}
%\newcommand*\glsr[2][]{\glsdisp[#1]{#2}{\glsentryshort{#2} (\glsentrylong{#2})}}
%\newcommand*\glsrpl[2][]{\glsdisp[#1]{#2}{\glsentryshortpl{#2} (\glsentrylongpl{#2})}}
%from documentation
%\newacronym[⟨key-val list⟩]{⟨label ⟩}{⟨abbrv ⟩}{⟨long⟩}
%above is short version of this
% \newglossaryentry{⟨label ⟩}{type=\acronymtype,
% name={⟨abbrv ⟩},
% description={⟨long⟩},
% text={⟨abbrv ⟩},
% first={⟨long⟩ (⟨abbrv ⟩)},
% plural={⟨abbrv ⟩\glspluralsuffix},
% firstplural={⟨long⟩\glspluralsuffix\space (⟨abbrv ⟩\glspluralsuffix)},
% ⟨key-val list⟩}

%\newacronym{api}{API}{Application Programming Interface }

\newacronym{fwhm}{FWHM}{Full Width at Half Maximum}
\newacronym{vhe}{VHE}{Very High Energy}
\newacronym{psf}{PSF}{Point Spread Function}
\newacronym[prefix={an~}]{nsb}{NSB}{Night-Sky Background}
\newacronym{dcr}{DCR}{Dark-Count Rate}
\newacronym{pe}{p.e.}{Photo-Electrons}
\newacronym{adc}{ADC}{Analogue-to-Digital Converter}
\newacronym{pde}{PDE}{Photon Detection Efficiency}
\newacronym{qe}{QE}{Quantum Efficiency}
\newacronym{spe}{SPE}{Single Photo-Electron}
\newacronym{eso}{ESO}{European Southern Observatory}
\newacronym{hawc}{HAWC}{High-Altitude Water Cherenkov Observatory}
\newacronym{iact}{IACT}{Imaging Atmospheric Cherenkov Telescope}
\newacronym{hess}{H.E.S.S.}{High Energy Stereoscopic System}
\newacronym{magic}{MAGIC}{Major Atmospheric Gamma Imaging Cherenkov Telescopes}
\newacronym{veritas}{VERITAS}{Very Energetic Radiation Imaging Telescope Array System}
\newacronym[prefixfirst={the~}]{cta}{CTA}{Cherenkov Telescope Array}
\newacronym{lst}{LST}{Large Size Telescope}
\newacronym{mst}{MST}{Medium Size Telescope}
\newacronym{sst}{SST}{Small Size Telescope}
\newacronym{sct}{SCT}{Schwarzschild-Couder Telescope}
\newacronym[prefixfirst={the~}]{gct}{GCT}{Gamma-ray Cherenkov Telescope}
\newacronym[prefixfirst={the~}]{chec}{CHEC}{Compact High Energy Camera}
\newacronym[prefixfirst={the~}]{chec-m}{CHEC-M}{\gls{chec} using \glspl{mapmt} as the detector}
\newacronym[prefixfirst={the~}]{chec-s}{CHEC-S}{\gls{chec} using \glspl{sipmt} as the detector}
\newacronym{pmt}{PMT}{Photomultiplier Tube}
\newacronym{mapmt}{MAPMT}{Multi-Anode Photomultiplier Tube}
\newacronym{sipmt}{SiPMT}{Silicon Photomultiplier Tube}
\newacronym{target5}{TARGET~5}{\gls{target} (version 5)}
\newacronym{targetc}{TARGET~C}{\gls{target} (version C)}
\newacronym{fpga}{FPGA}{Field-Programmable Gate Array}
\newacronym{adc2pe}{adc2pe}{conversion of \gls{adc} counts to \gls{pe}}
\newacronym{mc}{MC}{Monte-Carlo}
\newacronym{hv}{HV}{High Voltage}
\newacronym{impact}{ImPACT}{Image Pixel-wise fit for Atmospheric Cherenkov Telescopes}
\newacronym{fee}{FEE}{Front-End Electronics}
\newacronym{dc}{DC}{Direct Current}
\newacronym{ac}{AC}{Alternating Current}
\newacronym{rmse}{RMSE}{Root-Mean-Square Error}
\newacronym{oes}{OES}{Observation Execution System}
\newacronym{dpps}{DPPS}{Data Processing and Preservation System}
\newacronym{suss}{SUSS}{Science User Support System}
\newacronym{irf}{IRF}{Instrument Response Function}

\newglossaryentry{target}{type=\acronymtype, 
	name={TARGET}, 
	description={TeV Array Readout with GSa/s sampling and Event Trigger}, 
	first={TARGET (TeV Array Readout with GSa/s sampling and Event Trigger)}
}
\newglossaryentry{asic}{type=\acronymtype, 
	name={ASIC}, 
	description={Application-Specific Integrated Circuit}, 
	first={ASIC (Application-Specific Integrated Circuit)}
}
\newglossaryentry{vped}{type=\acronymtype, 
	name={Vped}, 
	description={Pedestal voltage input into the TARGET ASIC}, 
	first={input pedestal voltage (Vped)}
}
\newglossaryentry{fits}{type=\acronymtype, 
	name={FITS}, 
	description={Flexible Image Transport System}, 
	first={FITS (Flexible Image Transport System)}
}
\newglossaryentry{swig}{type=\acronymtype, 
	name={SWIG}, 
	description={Simplified Wrapper and Interface Generator}, 
	first={SWIG (Simplified Wrapper and Interface Generator)}
}
\newglossaryentry{tio}{type=\acronymtype, 
	name={TIO}, 
	description={Custom fits file format defined by TargetIO used for storing waveform data from TARGET-based cameras}, 
	first={TIO}
}
\newglossaryentry{astri}{type=\acronymtype, 
	name={ASTRI}, 
	description={Astrofisica con Specchi a Tecnologia Replicante Italiana}, 
	first={ASTRI (Astrofisica con Specchi a Tecnologia Replicante Italiana)}
}
\newglossaryentry{citiroc}{type=\acronymtype, 
	name={CITIROC}, 
	description={Cherenkov Imaging Telescope Integrated Read Out Chip}, 
	first={CITIROC (Cherenkov Imaging Telescope Integrated Read Out Chip)}
}
\newglossaryentry{aerie}{type=\acronymtype, 
	name={AERIE}, 
	description={Analysis and Event Reconstruction Integrated Environment}, 
	first={AERIE (Analysis and Event Reconstruction Integrated Environment)}
}
\newglossaryentry{corsika}{type=\acronymtype, 
	name={CORSIKA}, 
	description={COsmic Ray SImulations for KAscade}, 
	first={CORSIKA (COsmic Ray SImulations for KAscade)}
}
\newglossaryentry{kascade}{type=\acronymtype, 
	name={KASCADE}, 
	description={KArlsruhe Shower Core and Array DEtector}, 
	first={KASCADE (KArlsruhe Shower Core and Array DEtector)}
}

\newcommand{\glsf}[1]{\glsunset{#1}\glsentryfull{#1}}
\newcommand{\glsfpl}[1]{\glsunset{#1}\glsentryfullpl{#1}}
\newcommand*{\glsb}[2][]{\ifglsused{#2}{\acs[#1]{#2}}{%
 \glsunset{#2}%
 \acl[#1]{#2} [\acs[#1]{#2}]}}

% Nomenclature:
%\newglossaryentry{angelsperarea}{
%	name = $a$ ,
%	description = The number of angels per unit area,
%}
\makeglossaries

% To make text superscripts shortcuts
\renewcommand{\th}{\textsuperscript{th}} % ex: I won 4\th place
\newcommand{\nd}{\textsuperscript{nd}}
\renewcommand{\st}{\textsuperscript{st}}
\newcommand{\rd}{\textsuperscript{rd}}

\usepackage{xargs}
\usepackage[colorinlistoftodos,prependcaption,textsize=tiny]{todonotes}
\newcommandx{\change}[2][1=]{\todo[linecolor=red,backgroundcolor=red!25,bordercolor=red,#1]{#2}}
\newcommandx{\otherch}[2][1=]{\todo[linecolor=purple,backgroundcolor=purple!25,bordercolor=purple,#1]{#2}}
\newcommandx{\final}[2][1=]{\todo[linecolor=green,backgroundcolor=green!25,bordercolor=green,#1]{#2}}
\newcommandx{\notes}[2][1=]{\todo[linecolor=gray,backgroundcolor=gray!25,bordercolor=gray,#1]{#2}}
\newcommandx{\thiswillnotshow}[2][1=]{\todo[disable,#1]{#2}}

\usepackage{booktabs}

\usepackage{rotating}

\usepackage{listings}
\lstset{
	frameround=fttt,
	language=C++,
	breaklines=true,
	keywordstyle=\color{gray}\bfseries, 
	basicstyle=\ttfamily\color{gray},
	numberstyle=\color{black}
}
\lstset{ %
language=C++,                % choose the language of the code
basicstyle=\footnotesize,       % the size of the fonts that are used for the code
numbers=left,                   % where to put the line-numbers
numberstyle=\footnotesize,      % the size of the fonts that are used for the line-numbers
stepnumber=1,                   % the step between two line-numbers. If it is 1 each line will be numbered
numbersep=8pt,                  % how far the line-numbers are from the code
backgroundcolor=\color{white},  % choose the background color. You must add \usepackage{color}
showspaces=false,               % show spaces adding particular underscores
showstringspaces=false,         % underline spaces within strings
showtabs=false,                 % show tabs within strings adding particular underscores
frame=single,           % adds a frame around the code
tabsize=2,          % sets default tabsize to 2 spaces
captionpos=b,           % sets the caption-position to bottom
breaklines=true,        % sets automatic line breaking
breakatwhitespace=false,    % sets if automatic breaks should only happen at whitespace
escapeinside={\%*}{*)},          % if you want to add a comment within your code
xleftmargin=1.5em,
framexleftmargin=0.1em
}
\newcommand{\pkg}[1]{\texttt{#1}}
\newcommand{\cpp}{{C\nolinebreak[4]\hspace{-.05em}\raisebox{.4ex}{\tiny\bf ++}}}
\lstMakeShortInline[columns=fixed]|

\newcommand*\average[1]{\bar{#1}}
%\newcommand{\unit}[1]{\ensuremath{\, \mathrm{#1}}}

\usepackage{textgreek}
\usepackage{siunitx}
\DeclareSIUnit{\pe}{{p.e.}}
\sisetup{range-phrase=-, range-units=single}

\usepackage{float}
\usepackage[most]{tcolorbox}
%\tcbset{enhanced,colback=white,colframe=gray!75!black,fonttitle=\bfseries}
\newtcolorbox{requirement}[2][]{
    %float=htb,
    colback=white, 
    colframe=gray!75!black,
    every float=\centering,
    lower separated=false,
    fonttitle=\bfseries, 
    title={#2},
    #1
}
\newcommand{\requirementref}[1]{\textbf{#1}}

\usepackage{graphicx,wrapfig,lipsum}
\usepackage{subcaption}

\newcommand{\hangpara}{
 \setlength{\parindent}{0cm} % don't indent new paragraphs
 \hangindent=0.8cm % indent all subsequent lines
}

\graphicspath{
	{figures/}
	{figures/ch1/}
	{figures/ch2/}
	{figures/ch3/}
	{figures/ch4/}
	{figures/ch5/}
	{figures/ch6/}
	{figures/ch7/}
	{figures/ch8/}
	{figures/ch9/}
	{figures/a1/}
	{figures/a2/}
    {figures/a3/}
}

\usepackage{lineno}
\linenumbers
%\raggedbottom % Don't stretch line spacing to fill page

\begin{document}


%%%%% LINE SPACING
% OFFICIAL
%\setlength{\textbaselineskip}{22pt plus2pt}
% DRAFT
\setlength{\textbaselineskip}{18pt plus2pt minus1pt}


%%%%% LINE SPACING (roman-numbered pages)
\setlength{\frontmatterbaselineskip}{17pt plus1pt minus1pt}

% Leave this line alone; it gets things started for the real document.
\setlength{\baselineskip}{\textbaselineskip}


%%%%% SECTION NUMBERING DEPTH
% You have two choices.  First, how far down are sections numbered?  (Below that, they're named but
% don't get numbers.)  Second, what level of section appears in the table of contents?  These don't have
% to match: you can have numbered sections that don't show up in the ToC, or unnumbered sections that
% do.  Throughout, 0 = chapter; 1 = section; 2 = subsection; 3 = subsubsection, 4 = paragraph...
% The level that gets a number:
\setcounter{secnumdepth}{2}
% The level that shows up in the ToC:
\setcounter{tocdepth}{2}


%%%%% ABSTRACT SEPARATE
\begin{abstractseparate}
	The interaction of very-high-energy astrophysical gamma rays with the Earth's atmosphere produces extensive electromagnetic particle cascades. These showers of particles travel faster than the speed of light in air, and consequently emit photons of blue wavelength, known as Cherenkov radiation. The \gls{iact} technique enables the probing of the universe at \si{TeV} energies through the detection of these Cherenkov showers. The \gls{cta} will represent the next leap forward in gamma-ray astronomy, improving on the sensitivity of current \glspl{iact} by a factor of 10, encompassing energies from \SI{20}{GeV} to \SI{300}{TeV}, and operating as the first open observatory in this field. A major component of \gls{cta} is the \glspl{sst}, a necessary ingredient in exploring beyond the present energy frontier in gamma-ray astronomy.

One of three proposed designs for the \gls{sst} is the \gls{gct}. Utilising a dual-mirror Schwarzschild-Couder optical design, \gls{gct} enables a \SI{9}{\degree} \gls{fov} with a compact camera design. The camera developed for \gls{gct} is the \gls{chec}. Two prototypes for \gls{chec} have been built, each utilising different compact photosensor technology. \gls{chec-m} features \glspl{mapmt}, a pixelised extension of the \gls{pmt} technology extensively used by \glspl{iact}. \mbox{\gls{chec-s}} features \glspl{sipmt}, novel photosensors which utilise semiconductor technology for high-resolution photon counting over a large dynamic range. To fully utilise the signal output from these photosensors, and allow the opportunity for future data analysis procedures to be exploited, the signal received from these photosensors are digitised into waveforms following a trigger. These waveforms have a length of 96 samples with nanosecond precision.

In this thesis, the full calibration and signal-extraction pipeline currently adopted by \gls{chec} to reliably extract the Cherenkov signal from the waveforms is presented. The resulting performance of these procedures, and of the camera designs, is explored with respect to the requirements specified by the \gls{cta} Observatory. Potential improvements to the camera and calibration implementations are identified, and simulations of \gls{chec} are utilised to demonstrate the performance increase these proposals provide. Consequently, an improvement to the photosensor that would allow the \gls{chec-s} prototype design to comfortably meet the \gls{cta} requirements is specified. Testing of these improvements is anticipated to commence in early 2019. Finally, the results of the second on-telescope campaign for the \gls{chec-m} prototype are presented, during which observations of Cherenkov showers and optical measurements of Jupiter were conducted.
\end{abstractseparate}


\begin{romanpages}


%%%%% TITLEPAGE
\maketitle

%%%%% DEDICATION -- If you'd like one, un-comment the following.
%\begin{dedication}
%This thesis is dedicated to\\
%someone\\
%for some special reason\\
%\end{dedication}

%%%%% ACKNOWLEDGEMENTS
\begin{acknowledgements}
 	First and foremost, I would like to thank my supervisor, Prof Garret Cotter, for the immense amount of opportunities you have given me during my DPhil. I am eternally grateful for the support you have given me, while simultaneously letting me find my own route to develop into an independent researcher.

A major amount of thanks goes to Dr Richard White, for all the assistance you have provided me in the last four+ years, and for all the helpful and interesting discussions, even at odd times of the day. I really appreciate all the time you have invested in me, and all the patience you have shown me. It has all helped me improve and develop myself into a contributing, invested, and hopefully valuable member of the team.

I would like to thank Prof Jim Hinton for introducing me to CHEC and CTA, through the multiple summer placements you offered me while I studied for my MPhys degree at the University of Leicester. Those placements sparked my love for programming, and have led me down this prosperous academic path. Furthermore, I am grateful for the host at the Max-Planck-Institut für Kernphysik during 9~months of my DPhil. I make no secret about how much I loved my time there, and it truly advanced my proficiency as a researcher. I look forward to my continued interaction with MPIK.

A big thanks to all the GCT members, past and present, who enriched not only my work, but also my personal life. Specifically, I would like to thank Andrea De Franco, Tom Armstrong, and Heide Costantini. You three have been some of the biggest help I have from an early point in my DPhil. Furthermore, a big thanks to all those who took the time to proofread my thesis - Tom, Connor, Christoph, Paul, Rich and Garret. You were all a huge help and seriously reduced the stress I was under.

During my journey through undergrad, I met some amazing people who continue to give me immense amounts of happiness and support. Thank you Magda, Ben, Vili, and Viki. 

Mum and Dad, you have always believed in me, and you have always encouraged me to be my best. Thank you Mum for all the times we sat together revising for my exams, and thank you Dad for the incessant reminders to keep on top of my homework. 

Wiebke, you have been my biggest support during the writing of my thesis. I am so grateful for everything you do. Thank you so much for sacrificing so much of your time to proofread this work. Your advice and reassurances were probably the most valuable thing I received during the past few months. Thank you for being the best partner I could possibly ask for.

\end{acknowledgements}


%%%%% ABSTRACT
\begin{abstract}
	The interaction of very-high-energy astrophysical gamma rays with the Earth's atmosphere produces extensive electromagnetic particle cascades. These showers of particles travel faster than the speed of light in air, and consequently emit photons of blue wavelength, known as Cherenkov radiation. The \gls{iact} technique enables the probing of the universe at \si{TeV} energies through the detection of these Cherenkov showers. The \gls{cta} will represent the next leap forward in gamma-ray astronomy, improving on the sensitivity of current \glspl{iact} by a factor of 10, encompassing energies from \SI{20}{GeV} to \SI{300}{TeV}, and operating as the first open observatory in this field. A major component of \gls{cta} is the \glspl{sst}, a necessary ingredient in exploring beyond the present energy frontier in gamma-ray astronomy.

One of three proposed designs for the \gls{sst} is the \gls{gct}. Utilising a dual-mirror Schwarzschild-Couder optical design, \gls{gct} enables a \SI{9}{\degree} \gls{fov} with a compact camera design. The camera developed for \gls{gct} is the \gls{chec}. Two prototypes for \gls{chec} have been built, each utilising different compact photosensor technology. \gls{chec-m} features \glspl{mapmt}, a pixelised extension of the \gls{pmt} technology extensively used by \glspl{iact}. \mbox{\gls{chec-s}} features \glspl{sipmt}, novel photosensors which utilise semiconductor technology for high-resolution photon counting over a large dynamic range. To fully utilise the signal output from these photosensors, and allow the opportunity for future data analysis procedures to be exploited, the signal received from these photosensors are digitised into waveforms following a trigger. These waveforms have a length of 96 samples with nanosecond precision.

In this thesis, the full calibration and signal-extraction pipeline currently adopted by \gls{chec} to reliably extract the Cherenkov signal from the waveforms is presented. The resulting performance of these procedures, and of the camera designs, is explored with respect to the requirements specified by the \gls{cta} Observatory. Potential improvements to the camera and calibration implementations are identified, and simulations of \gls{chec} are utilised to demonstrate the performance increase these proposals provide. Consequently, an improvement to the photosensor that would allow the \gls{chec-s} prototype design to comfortably meet the \gls{cta} requirements is specified. Testing of these improvements is anticipated to commence in early 2019. Finally, the results of the second on-telescope campaign for the \gls{chec-m} prototype are presented, during which observations of Cherenkov showers and optical measurements of Jupiter were conducted.
\end{abstract}


%%%%% MINI TABLES OF CONTENTS
\dominitoc


% This aligns the bottom of the text of each page.  It generally makes things look better.
\flushbottom


%%%%% TABLE OF CONTENTS
\tableofcontents


%%%%% LIST OF FIGURES
\listoffigures
	\mtcaddchapter


%%%%% LIST OF ABBREVIATIONS
%\include{text/abbreviations}
\newpage
\printglossary[nonumberlist,type=\acronymtype,title=Abbreviations]


\end{romanpages}


%%%%% CHAPTERS
\flushbottom
\adjustmtc
%  \chapter{\label{ch1-intro}Introduction} 

\minitoc

\section{Cherenkov Radiation}

\begin{figure}
  \begin{subfigure}[b]{0.49\textwidth}
    \includegraphics[width=\textwidth]{dipole_slow}
    \caption{$v < \frac{c}{n}$}
    \label{fig:dipole_slow}
  \end{subfigure}
  \hfill
  \begin{subfigure}[b]{0.49\textwidth}
    \includegraphics[width=\textwidth]{dipole_fast}
    \caption{$v \ge \frac{c}{n}$}
    \label{fig:dipole_fast}
  \end{subfigure}
  \caption[Polarisation produced in a dielectric medium due to the presence of a charged particle.]{Polarisation produced in a dielectric medium due to the presence of a charged particle, for the cases of a subluminal and superluminal particle.}
\end{figure}

\begin{figure}
	\centering\includegraphics[width=0.5\textwidth]{cherenkov_geom} 
	\caption[Geometry of the wavefronts involved in Cherenkov radiation production.]{Geometry of the wavefronts involved in Cherenkov radiation production. The particle travels at a greater speed than the wavefronts propagate.}
	\label{fig:cherenkov_geom}
\end{figure}

When a charged particle moves slowly through a dielectric medium, the electric field of the particle distorts the nearby atoms. Momentarily, these atoms are transformed into elementary dipoles where the charged particles that constitute the atom are arranged with respect to the electric field of the travelling particle (Figure~\ref{fig:dipole_slow}). Due to the complete symmetry of this polarisation around the travelling particle, no net field is produced by the dielectric medium. However, if instead the velocity of the charged particle is faster than the speed light travels in that medium, an asymmetry along the particle trajectory is formed in the polarisation of the surrounding atoms (Figure~\ref{fig:dipole_fast}), resulting in a net dipole field. As the particle continues through the medium, elements of the polarised medium will release a brief burst of electromagnetic radiation. Generally these electromagnetic waves interfere destructively, except inside in the forward direction along the particles trajectory in an opening angle $\theta$. Although the full characterisation of this relativistic effect is complex, a simple consideration of the geometry involved, shown in Figure~\ref{fig:cherenkov_geom}, can be used to describe $\theta$ \cite{Jelley1958a}. In a time $\Delta t$ a particle travels a distance $\beta c \Delta t$ where $\beta = \frac{v}{c}$, while the emitted light will travel a distance $\frac{c}{n} \Delta t$ in a medium with refractive index $n$. This results in the relation:
\begin{equation} \label{eq:cherenkov_angle}
\cos \theta = \frac{c}{vn}.
\end{equation}
The blue light emitted in this constrained opening angle, via this phenomena, is known as Cherenkov radiation.

\section{Atmospheric Cherenkov Showers} \label{section:cherenkov_shower_intro}

\begin{figure}
	\centering\includegraphics[width=\textwidth]{cascade} 
	\caption[Production of a extended electromagnetic particle cascade.]{Production of a extended electromagnetic particle cascade, demonstrating the different components and interactions.}
	\label{fig:cascade}
\end{figure}

The Earth's atmosphere is effectively opaque to photon energies above \SI{10}{eV} \cite{Weekes2003}. To perform astronomy at higher energies, one must usually leave the Earth's atmosphere. However, at energies above \SI{\ge 10}{GeV}, a ``gamma-ray window'' exists where astronomy can be performed using the Cherenkov radiation produced by the cascade of particles resulting from the interaction between the gamma ray and the atmosphere.

Two electromagnetic interactions are responsible for the creation of this cascade:
\begin{description}
\item [Pair Production] The conversion of a photon into an electron-positron pair in the presence of an atom (such as an atmospheric particle). The energy of the photon must exceed the sum of the rest masses of an electron and positron (\SI{1.022}{MeV}). The electron-positron pair share the energy of the progenitor photon, and continue on a similar trajectory. This is the dominating interaction process for photons above \SI{\ge 10}{MeV} \cite{Weekes2003}.
\item [Bremsstrahlung radiation] The emission of a photon due to the interaction of a charged particle with the electric field on an atom (such as an atmospheric particle). This process allows further gamma rays to be produced.
\end{description}
The interplay between these two processes, occurring after each transversal of a radiation length, produces the extensive cascade of energetic electromagnetic particles. This is illustrated in Figure~\ref{fig:cascade}. The charged particles produced by the pair production in this cascade are responsible for the generation of the Cherenkov light. This cascade is often known as a ``Cherenkov shower''.

This cascade continues until the ionisation energy losses are equal to the radiation losses. This number of remaining particles after this point, known as the ``shower maximum'' begins to diminish. For a \SI{1}{TeV} shower this occurs at \SI{\sim 8.4}{km} altitude \cite{Weekes2003}. The produced Cherenkov light is collimated along the progenitor gamma ray trajectory, and produces a pool of blue light on the ground, with a radius of \SI{\sim 120}{m} \cite{Hillas1996a}. If the direction of the Cherenkov shower is extrapolated back to the cosmic sphere, the location of the source that produced the gamma ray can be inferred. Although the amount of energy that goes into Cherenkov photon production is a tiny fraction of the total energy, the atmosphere acts as a consistent calorimeter, therefore allowing an accurate reconstruction of the progenitors energy from the amount of Cherenkov photons produced. 

A further characteristic of the Cherenkov shower is the time profile. The entire shower typically lasts \SI{\sim 5}{ns}. Therefore, despite the abundance of showers in the sky, and the visible wavelength of the Cherenkov light, they are imperceivable by the human eye. Furthermore, due to the superluminal velocities of the particles inside the cascade, the last Cherenkov photons produced at the end of the shower reach the ground before the first Cherenkov photons produced at the start of the shower. With different sections of the showers arriving at different times, the Cherenkov shower measurements display a time gradient across the image. 

\section{Imaging Atmospheric Cherenkov Telescopes}

A primary issue in \gls{vhe} astronomy is the low flux (${\sim} 0.2$ per \si{m \squared} per year \cite{Franco2016}), requiring a collection area that is not feasible for space telescopes. If instead the Cherenkov showers are used to detect the gamma rays, large arrays of optical telescopes can be built to provide stereoscopic imaging of the Cherenkov showers. These telescopes are known as \glspl{iact}. The multiple stereoscopic views of individual showers provided by arrays of \glspl{iact} allow accurate reconstruction of the properties of the shower, such as direction and energy. The topic of reconstruction is discussed in Chapter~\ref{ch6-reduction}.

As the \gls{iact} technique involves imaging the Cherenkov showers, which are much larger than typical astronomy targets, \glspl{iact} do not require the resolving power of typical optical telescopes. Instead the priorities of an \gls{iact} optical system are to maximise: 
\begin{itemize}
\item Mirror collection area, such that more photons can be collected. This enables fainter showers to be detected, thereby lowering the energy threshold.
\item \gls{fov}, which improves the surveying capabilities and eases the study of extended sources.
\end{itemize}
Furthermore, the large collection area provided by the light pool of the Cherenkov shower enables a modest telescope to still make a large amount of gamma ray detections, enabling this technique to be viable despite the small flux.

\begin{figure}
	\centering\includegraphics[width=\textwidth]{nsb} 
	\caption[Comparison of Cherenkov and NSB spectrum.]{Comparison of Cherenkov and NSB spectrum. The Cherenkov spectrum shown is expected at an altitude of \SI{2200}{m}. The NSB spectrum shown was measured at La Palma \cite{Bouvier2013}.}
	\label{fig:nsb}
\end{figure}

Two major background components need to be accounted for in \glspl{iact}:
\begin{description}
\item [Cosmic Ray Background] Protons (and heavier hadronic nuclei) are also capable of producing Cherenkov showers that are not entirely dissimilar to electromagnetic showers. As these particles are charged, they have been deflected by interstellar magnetic fields on their journey from their source, and therefore are not useful for \glspl{iact}. Therefore these showers provide an isotropic background which is 1,000 times as numerous than the shower rate received from the discreet gamma-ray sources. However, a hadronic shower exhibits a morphology that is broader and less symmetric than that obtained from gamma-ray showers. Additionally, distinct features such as ``muon rings'', produced by highly penetrating muons reaching low altitudes such that the full Cherenkov cone is visible in a single telescope, accompany hadronic showers. Parametrisations of the Cherenkov shower image therefore enable the discrimination of the hadronic showers (see Chapter~\ref{ch6-reduction}. \change{images?}
\item [Night Sky Background] Due to the optical sensitivity of the cameras used by \glspl{iact}, the measurements taken are susceptible to starlight, moonlight, and artificial light pollution. The \gls{nsb} spectrum for La Palma, compared to the expected Cherenkov spectrum at an altitude of \SI{2200}{m}, is displayed in Figure~\ref{fig:nsb}. This background is excluded in two ways. Firstly, smart trigger logic and strict thresholds (such as the one described in Chapter~\ref{ch2-mechanics}) eliminate the triggering on \gls{nsb} photons. Secondly, unbiased charge extraction technique (described in Chapter~\ref{ch6-reduction}) exclude this noise from the signal.
\end{description}

The application of the \gls{iact} technique was first attempted in the 1960s, but the first large optical reflector built with the purpose of gamma-ray astronomy was the Whipple 10 m telescope in southern Arizona, 1968. At first, gamma-ray astronomy was polluted with unsubstantial claims of transient signals from a variety of pulsars and binaries, but these signals had marginal statistical significance \cite[][p.~9]{Weekes2003}. It wasn't until 20 years later, after further development of the technique, that the Crab Nebula was detected by Whipple in 1989, thus reigniting interest in the development of gamma-ray astronomy.

\begin{figure}
  \centering
  \begin{subfigure}[b]{0.35\textwidth}
  \includegraphics[width=\textwidth]{hess}
  \caption{H.E.S.S.}
  \label{fig:hess}
  \end{subfigure}
  ~
  \begin{subfigure}[b]{0.35\textwidth}
  \includegraphics[width=\textwidth]{magic}
  \caption{MAGIC}
  \label{fig:magic}
  \end{subfigure}
  ~
  \begin{subfigure}[b]{0.45\textwidth}
  \includegraphics[width=\textwidth]{veritas}
  \caption{VERITAS}
  \label{fig:veritas}
  \end{subfigure}
  \caption{Photos of modern \glspl{iact}.}
  \label{fig:iacts}
\end{figure}

\begin{figure}
	\centering\includegraphics[width=\textwidth]{sensitivity} 
	\caption[Differential sensitivity of CTA.]{Differential sensitivity of CTA predicted by Monte Carlo simulations, compared to the performance of other gamma-ray instruments \cite{cta-performance}. The differential sensitivity has been defined as the minimum flux needed by CTA to obtain a 5-standard-deviation detection of a point-like source.}
	\label{fig:sensitivity}
\end{figure}

Modern \glspl{iact} include \gls{magic}, \gls{veritas}, and the most recent, \gls{hess} (Figure~\ref{fig:iacts}). All three of these telescope systems operate with the advantage of stereoscopic collaboration. In order to improve on the current \glspl{iact}, an array of ${\sim} 100$ telescopes was proposed, called the \gls{cta}. This array will have \cite{Acharya2013}:
\begin{itemize}
\item an improved sensitivity of 10 times over previous \glspl{iact} (Figure~\ref{fig:sensitivity}),
\item an observable gamma-ray energy range of \SI{30}{GeV} to \SI{100}{TeV},
\item a large (\SI{\sim 8}{\degree}) field of view for surveys,
\item improved angular and energy resolution,
\item and will be the first \gls{iact} to operate as an open observatory.
\end{itemize}

\section{The Cherenkov Telescope Array}

\begin{figure}
	\centering\includegraphics[width=\textwidth]{sensitivity_tel} 
	\caption[Differential sensitivity of the different CTA telescope types.]{Contribution of each telescope type within \gls{cta} to the total differential sensitivity \cite{Marano2014}.}
	\label{fig:sensitivity_tel}
\end{figure}

\gls{cta} will consist of three different sized telescopes:
\begin{itemize}
\item The \gls{lst}, with a mirror diameter of about \SI{23}{m} to enable the collection of as many photons as possible from the low energy showers (\SIrange{20}{150}{GeV}).
\item The \gls{mst}, covering the mid-range \SIrange{0.1}{10}{TeV} with mirror diameters of \SI{12}{m}.
\item The \gls{sst}, monitoring the high energies of \SIrange{1}{300}{TeV}, with mirror diameters of around \SI{4}{m}. Due to the rarity of higher energy showers, many \glspl{sst} need to be spread over an area of several square kilometres, to increase the chance of a detection \cite{Acharya2013}.
\end{itemize}
To illustrate these descriptions, the contributions of each telescope to the sensitivity of \gls{cta} is demonstrated in Figure~\ref{fig:sensitivity_tel}.

Two \gls{cta} sites will exist. A northern hemisphere site for extragalactic observations will be built at La Palma, and is planned to contain 4 \glspl{lst} and 16 \glspl{mst}. As this site will focus on the energy range from \SI{20}{GeV} to \SI{20}{TeV}, no \glspl{sst} are included on the northern site. A southern hemisphere site will provide observations of the galactic plane, spanning the full energy range of \gls{cta}. Planned to be built nearby the Paranal Observatory in the Atacama Desert in Chile, the southern array is planned to feature 4 \glspl{lst}, 15 \glspl{mst}, and 70 \glspl{sst}, spread over \SI{4}{km \squared}.

\section{Small Sized Telecopes}

Three designs for an \gls{sst} have been proposed:
\begin{itemize}
\item 
\end{itemize}


There exists 3 designs of the \gls{sst}: SST-1M (a single mirror design), ASTRI and \pgls{gct} (both being dual-mirror designs). These are shown in \autoref{fig:sst}. I am part of the \pgls{gct} group, which consists of the France-based GATE structure and mirrors, and the UK-based \gls{chec}. It is the development and commissioning of \gls{chec} that is the focus of my D.Phil. 

- Science with ssts
- Three designs

\section{Gamma-ray Cherenkov Telescope}

- what makes us better - lightweight telescope, full waveform readout
- optics








\notes[inline,caption={}]{
	\section{Plan}
	\subsection{Topics}
	\begin{itemize}
		\item High Energy Astrophysics
		\begin{itemize}
			\item Fermi
			\item Fermi Bubbles
			\item HAWC
		\end{itemize}
		\item IACTs
		\item CTA
		\item CTA Science
		\begin{itemize}
			\item Science Cases
			\item Use "Science with CTA" paper
		\end{itemize}
		\item SSTs
		\item SST Science
		\begin{itemize}
			\item What do we contribute?
			\item What can't be done without us?
		\end{itemize}
		\item GCT
		\item CHEC
		\begin{itemize}
			\item What makes us better?
			\item Advantages of Schwarzchild-Couder
			\begin{itemize}
				\item Increased FoV
				\item Size
				\item Cost
			\end{itemize}
			\item Advantages of full waveform readout
			\item Other Advantages?
			\begin{itemize}
				\item Trigger
				\item Energy/power/voltage Requirements
				\item Commonalities (SCT)
			\end{itemize}
		\end{itemize}
	\end{itemize}
	\subsection{Questions}
	\begin{itemize}
		\item ?
	\end{itemize}
}

\change[inline]{Shower properties, photons from bottom of shower are received before those at the top as the particle travels faster than light. Good figure in \cite{Cogan2006}}
\change[inline]{Gamma/Hadron/Lepton}

\change[inline]{Terminology note: charge not used in terms of columns, it refers to counts of photoelectrons, for which mV and ADC are a proxy of. Do a ctrl-f at end to check how charge is used}

\change[inline]{trigger efficiency???}

\change[inline]{introduce SC optics with reference to "modern iteration for utilisation in Cherenkov shower described by \cite{Vassiliev2007}" Giro2017}

low number of photons above 50 TeV, very low, frontier

\section{Atmospheric Cherenkov Showers}

\section{Imaging Atmospheric Cherenkov Technique}

\subsection{Cherenkov Shower Characteristics}

\subsection{Photon Arrival Time} \label{section:photon_arrival_time}

\change[inline]{Quoted from Holder2005: The longitudinal development of an air shower is reflected in the long axis of the elliptical image recorded in the camera. The photon arrival time profile along this axis is largely a result of geometrical path length differences, and hence the shower core distance. As the shower particles move faster than the speed of light in air, when the shower has a small core distance Cherenkov light emitted from lower in the atmosphere is received at the telescope first. At large core distances, this situation is reversed, as the Cherenkov light travel time from the shower to the telescope dominates. The effect of this is to produce a timing gradient along the long axis of the image, the size and sign of which depend upon the core distance. For gamma-ray showers from a point source at the centre of the field of view, the shower core distance is directly related to the angular distance in the camera of the image from the source position.
Figure}

\section{The Cherenkov Telescope Array}

CTA will be, for the first time in VHE gamma-ray astronomy, operated as an open observatory.

\section{Small Sized Telescopes}

\section{Science with the Small Sized Telescopes}


%  \chapter{\label{ch2-mechanics}Camera Design \& Mechanics}

\minitoc

\notes[inline,caption={}]{
	\section{Plan}
	\subsection{Topics}
	\begin{itemize}
		\item Introduce TARGET architecture \& Wilkinson ADC
		\item Different TARGET versions
		\item FEE
		\item MAPMs
		\item SiPMS
		\begin{itemize}
			\item How they work
			\item Comparison investigations
			\item Property trade-offs
		\end{itemize}
		\item CHEC-M
		\item Changes for CHEC-S
		\item Future - MUSIC ASICs
	\end{itemize}
	\subsection{Questions}
	\begin{itemize}
		\item ?
	\end{itemize}
}

\section{Introduction}


\begin{figure}
	\centering\includegraphics[width=\textwidth]{figures/images/target5diagram} 
	\caption[Functional block diagram of the TARGET~5 ASIC.]{Functional block diagram of the TARGET~5 ASIC \cite{Albert2017} \change[inline]{Add more details}}
	\label{fig:target5diagram}
\end{figure}

%  \chapter{\label{ch3-architecture}CTA Architecture} 

\minitoc

\notes[inline,caption={}]{
	\section{Plan}
	\subsection{Topics}
	\begin{itemize}
		\item Requirements
		\item Data Levels
	\end{itemize}
	\subsection{Questions}
	\begin{itemize}
		\item ?
	\end{itemize}
}

\section{Introduction}

\section{Requirements}

test

\section{Data Levels}
  \chapter{\label{ch4-software}Software} 

\minitoc

\notes[inline,caption={}]{
	\section{Plan}
	\subsection{Topics}
	\begin{itemize}
		\item TargetIO/TargetDriver
		\item TargetCalib
		\item ctapipe
		\item gammapy/CTOOLS
	\end{itemize}
	\subsection{Questions}
	\begin{itemize}
		\item ?
	\end{itemize}
}

\section{Introduction}

In Chapter~\ref{ch3-architecture}, the data processing steps that are required in order to obtain the science data that is released by the \gls{cta} Observatory are described. In order to go from camera trigger to science results, a number of software packages must be developed, and be provided to \gls{cta} as in-kind contributions, as instructed by the \gls{cta} Architecture. 

This chapter provides an outline of the software packages used in the pipeline for \gls{chec} and \gls{cta}, alongside my contributions to them.

\section{TARGET Libraries}

A collection of libraries have been created to operate, readout, and calibrate the cameras containing \gls{target} modules (\gls{chec} and the \gls{sct} camera), and are therefore known as the ``TARGET Libraries''. These low-level libraries are wrote in \cpp as they prioritise efficiency over flexibility. To enable the use of these libraries from the Python packages used in waveform reduction, a Python wrapper for these libraries is automatically generated during compilation by \gls{swig}\footnote{http://www.swig.org/}.

These libraries are presently stored on the \gls{cta}-SVN version control server, and installation instructions can be found at \url{https://forge.in2p3.fr/projects/gct/wiki/Installing_CHEC_Software}, provided you have permissions to the \gls{gct} Redmine.

\subsection{\pkg{TargetDriver}}

In order to operate, or ``drive'', the \gls{target} modules, the \pkg{TargetDriver} library is required. This \cpp library configures the TARGET modules, and listens for the UDP packets containing the waveform data.

\subsection{\pkg{TargetIO}}

\begin{figure}
  \centering
  \includegraphics[width=\textwidth]{tio}
  \captionsetup{singlelinecheck=off}
  \caption[Simple overview of the data-flow for waveform samples within the TARGET libraries]{Simple overview of the data-flow for waveform samples within the TARGET libraries:
  \begin{enumerate}[label={\arabic*)}]
  \item 8-bit/char packets are sent from the TARGET FPGA and stored directly to file. A waveform sample is 12-bit, therefore the first four bits of the first 8-bit sample packet are used to indicate sample order.
  \item When reading a sample from the R0 TIO file, the first four bits are ignored, and the remaining twelve bits are combined into an unsigned 16-bit sample. The samples are passed to \pkg{TargetCalib} for calibration. The resulting calibrated floating-point sample is scaled and offset to fit into an unsigned 16-bit integer for storage.
  \item When reading a sample from a R1 TIO file, the entirety of the two 8-bit packets are kept and combined. The value is returned to floating-point format using the OFFSET and SCALE stored in the file header.
  \end{enumerate}
  
  Although only the samples are shown here, all other waveform data is also sent along this stream, including ASIC and Channel number, indicating the start of a new waveform. 
  }
  \label{fig:tio}
\end{figure}

The file format used to store waveforms from \gls{target} modules is a custom \gls{fits} format defined by \pkg{TargetIO}, hereby referred to as the \gls{tio} format. This library is always used to read and write waveform and header (information such as observation time) data to the \gls{tio} file. \gls{tio} files can either contain \textit{R0} (uncalibrated)  or \textit{R1} (low-level calibrated) waveform data. Each sample in a waveform is stored to file as an unsigned 16-bit integer. The raw waveform digital counts measured by the camera are serialised in an unsigned 12-bit integer format, and the data packets received from the \gls{target} \gls{fpga} are 8-bit in size, therefore the first 4 bits of a waveform sample inside an \textit{R0} \gls{tio} file are not used. When storing calibrated waveforms, the post-calibration floating-point sample is scaled and offset to fit the full 16-bit unsigned integer. These scale and offset values are stored in the file header and automatically applied to convert the sample back into floating-point format when read. Figure~\ref{fig:tio} demonstrates the data-flow processes involving the \gls{tio} files.

To ensure the full efficiency of the \cpp library is exploited via the Python wrapper, I contributed the |WaveformArrayReader| class, which, when passed a contiguous block of memory (such as a \lstset{language=Python}|numpy.array|), promptly fills the array with the entire camera's waveform data for that event. For example, to read an \textit{R1} \gls{tio} file from Python:

\begin{lstlisting}[language=Python]
import numpy as np
from target_io import WaveformArrayReader

# Create the reader and get the number of pixels and number of samples from the header
reader = WaveformArrayReader("/path/to/file/Run17473_r1.tio")
n_pixels = reader.fNPixels
n_samples = reader.fNSamples

# Generate the memory to be filled in-place
waveforms = np.zeros((n_pixels, n_samples), dtype=np.float32)
first_cell_ids = np.zeros(n_pixels, dtype=np.uint16)  # Storage cell id for the first sample of the event per pixel

# Fill the arrays
event_index = 20
reader.GetR1Event(event_index, waveforms, first_cell_ids)
# `waveforms' array is now filled with entire event's waveform data
\end{lstlisting}

\subsection{\pkg{TargetCalib}}

To correct for the effects of the \gls{target} electronics on the waveforms, \pkg{TargetCalib} was built. I have led the development of this package since its early development. The calibrations performed by this library are detailed in Chapter~\ref{ch5-calibration}. This package has also been adopted by \gls{sct} recently.

The main classes in the library include:

\lstset{language=C++}
\begin{description}
\item [\textbf{PedestalMaker}] Generates the \textit{Pedestal} calibration file.
\item [\textbf{TfMaker}] Generates the \textit{Transfer Function} calibration file.
\item [\textbf{Calibrator}] Applies the aforementioned calibration files to the waveform samples.
\item [\textbf{Mapping}] Handles the files containing the camera's pixel mapping, and provides an interface to the information. This class is necessary due to the non-intuitive mapping between physics pixel position, and order of pixel readout (Figure~\change{figure showing the pixel positions, camera or module?}). Most commonly, this mapping is used for the plotting of camera images. The class is compatible with the mapping of any square-pixel telescope, and customisable to provide the mapping of the pixels in a single module, the mapping of the superpixels, the mapping of the modules, or the neighbours to a pixel/superpixel/module. This class will be deprecated once the central \gls{cta} database of telescope configurations exists.
\item [\textbf{CameraConfiguration}] Provides an interface to certain camera-version dependant variables. Currently the variables that might change with camera-version (stored in the \gls{tio} file header) include number of storage cells, pixel mapping, and reference pulse shape. The correct version of the parameter is returned according to the camera-version provided, allowing for the automated processing of the data of different camera versions. This class will also be replaced by the central \gls{cta} database.
\end{description}

Efforts are being made to improve the \pkg{TargetCalib}'s (more specifically the |Calibrator| class's) efficiency in terms of both memory and processing time, as it will need to meet the \gls{cta} Requirements for \textit{Online Analysis} (Chapter~\ref{ch3-architecture}). It is possible that in the future there will be two separate |Calibrator| classes for the \textit{Online} and \textit{Offline Analyses} respectively.

\section{Reduction Tools}

Tools used to process the waveforms in order to either characterise the camera or progress down the data-level-chain (Figure~\ref{fig:dataflow}) are often referred to as ``reduction tools''. Within the \gls{chec} group we utilise Python for all of our waveform reduction. We made this choice due to its high popularity for data science and signal processing and its extensive library of statistical and numerical packages. The most important examples of these packages include:

\begin{description}
\item [\textbf{NumPy\footnotemark}] \footnotetext{http://www.numpy.org/} Enables the efficient processing of numerical data. This is accomplished using their powerful N-dimensional array object known as a |numpy.array|. At the lowest level, a |numpy.array| is a contiguous block of memory much like a C array. However, NumPy defines many statistical methods which utilise optimised low-level C and Fortran operations to process the contained data in the most efficient way possible, often performing better than handwritten C or Fortran.
\end{description}

Different reduction packages may be designed with different purposes, but each can potentially import methods from another, which is especially possible when developing in Python. Many other \gls{cta} groups have also adopted Python for their waveform reduction software, but it is not a standard across \gls{cta}.

\subsection{\pkg{ctapipe}}

\subsection{\pkg{CHECLabPy}}

\section{Science Tools}

\subsection{\pkg{GammaPy}}

\subsection{\pkg{CTOOLS}}
%  \chapter{\label{ch5-calibration}Calibration} 

\minitoc

\lstset{language=Python}

\section{Introduction}

In order to obtain meaningful and reliable results from the camera, a number of calibrations must be applied to the waveforms read. A primary objective of my DPhil was to investigate the most optimal and efficient approaches for these calibrations (in accordance with the \gls{cta} requirements described in Chapter~\ref{ch3-architecture}), and to determine if additional calibrations are required.

When I joined the \gls{chec} development, the calibration discussion was still in its infancy. Some approaches had been tested in a laboratory environment \cite{Bechtol2012}, but there had been little discussion on how exactly the calibrations could be applied efficiently in an analysis pipeline, where one might not be able to use the same detailed calibration due to limited resources (such as memory and processing time). A major contribution of my DPhil was to prototype the calibration procedures, develop an approach for a calibration pipeline, write the software to perform such a pipeline, and finally assess its performance. This was an iterative process, the development of which is still ongoing. However, a procedure now exists that allows us to obtain meaningful results from the waveform data, a capability that is of paramount importance in the commissioning of the camera.

In this chapter I will outline each of the calibration steps that are presently adopted for \gls{chec}. They are introduced in the general order that they are applied, and split into the categories of \gls{target} \gls{asic}, photosensor, and "other" calibrations.

\section{TARGET Calibration}

The calibrations described in this section relate to the \gls{target} module. As detailed in Chapter~\ref{ch2-mechanics}, the \gls{target} \gls{asic} is responsible for the sampling, digitisation and readout of the waveform data. As a result, there are two calibrations that are solely related to the \gls{target} \gls{asic}: electronic pedestal subtraction and the linearity correction via the transfer function. 

The functional block diagram of the \gls{target} \gls{asic} in Figure~\ref{fig:target5diagram} outlines the electronics that require calibration, and can be used as a reference in the following descriptions.

As the calibrations in this section are very low-level, and related to \gls{chec}'s specific \gls{fee}, they are handled by the TargetCalib library (Chapter~\ref{ch4-software}).

\subsection{Electronic Pedestal Subtraction}

\begin{figure}
\begin{minipage}[t]{.49\textwidth}
  \centering
  \includegraphics[width=0.92\textwidth]{rawwf_10} 
  \captionof{figure}[Raw waveform]{TARGET-C waveform as read out from CHEC-S, showing the electronic pedestal in the absence of any other input, before any calibration is applied.}
  \label{fig:rawwf}
\end{minipage}%
\hfill
\begin{minipage}[t]{.49\textwidth}
  \centering
  \includegraphics[width=\textwidth]{pulse_raw_vs_pedestal}
  \captionof{figure}[Comparison of pedestal-subtracted waveform with raw waveform]{CHEC-S waveform  containing a \SI{5}{\pe} pulse, before and after pedestal subtraction.}
  \label{fig:pulse_raw_vs_pedestal}
\end{minipage}
\end{figure}

\begin{figure}
  \begin{subfigure}[b]{0.49\textwidth}
    \includegraphics[width=\textwidth]{r0_cherenkov_image_mirrored_cropped}
    \caption{Raw image.}
    \label{fig:r0_cherenkov_image_mirrored_cropped}
  \end{subfigure}
  \hfill
  \begin{subfigure}[b]{0.49\textwidth}
    \includegraphics[width=\textwidth]{r1_cherenkov_image_mirrored_cropped}
    \caption{Pedestal-subtracted image.}
    \label{fig:r1_cherenkov_image_mirrored_cropped}
  \end{subfigure}
  \centering
  \begin{subfigure}[b]{0.49\textwidth}
    \includegraphics[width=\textwidth]{r1pe_cherenkov_image_mirrored_cropped}
    \caption{Final calibrated image.}
    \label{fig:r1pe_cherenkov_image_mirrored_cropped}
  \end{subfigure}
  \caption[Comparison of calibration stages with a Cherenkov shower image.]{The same image of a Cherenkov shower taken with CHEC-M, but at different stages of calibration: (a) An image of the sample values taken at a timeslice corresponding to the shower maximum. (b) The same timeslice, but after the pedestal subtraction calibration has been applied to the samples. (c) The final image, after charge extraction (Chapter~\ref{ch6-reduction}) and calibration into units of photoelectrons. The value of the samples An integration window was chosen using the \textit{Neighbour Peak Finding} technique (Chapter~\ref{ch6-reduction}) on the \si{\pe} calibrated waveforms. The same samples were then integrated for each of the calibration stages.}
\end{figure}

The most important, but also the simplest, waveform data calibration to apply is the subtraction of the electronic pedestal. Each cell in the storage array of the \gls{asic} is a unique capacitor. For a specific \gls{vped}, each capacitor has its own resulting electronic pedestal value. As each sample of the waveform corresponds to a single storage cell, each sample therefore has a unique pedestal value to be subtracted. This is apparent in Figures~\ref{fig:rawwf}~and~\ref{fig:pulse_raw_vs_pedestal} where the variation from sample-to-sample is very large in the raw waveform, and the low-amplitude pulses are almost indistinguishable. The fluctuations in the raw waveforms between pixels is also significant, to the point where low-amplitude Cherenkov showers are undetectable in the camera (Figure~\ref{fig:r0_cherenkov_image_mirrored_cropped}). However, the dominating variations are between \glspl{asic}. As a result, the outlines of the \glspl{asic} are the dominating feature in camera images containing raw samples, such as Figure~\ref{fig:r0_cherenkov_image_mirrored_cropped}. With a pedestal-subtraction calibration alone, the waveforms are transformed into a state in which a moderate amount of Cherenkov shower assessment can be performed, as demonstrated in Figure~\ref{fig:r1_cherenkov_image_mirrored_cropped}.

\begin{figure}
	\centering
    \includegraphics[width=\textwidth]{cellwf_15} 
	\caption[Storage-cell-amplitude dependence on position in the waveform.]{Average amplitude of the electronic pedestal for a single storage cell in a TARGET-C ASIC, at different positions in the waveform. Error bars indicate the standard deviation of the amplitudes. The grey dashed lines indicate the position of the block edges in the waveform for this cell. The average of the values inside each block segment equals the pedestal value stored in the lookup table for that cell, in each of those block positions.} 
	\label{fig:cellwf}
\end{figure}

There are $2^{14} = 16,384$ storage cells per channel (for \gls{chec-m}, $2^{12} = 4096$ for \gls{chec-s}), therefore one could naively conclude that there are $32 (Modules) \times 64 (Channels) \times 16,384 (Cells)$ pedestal values to keep record of. However, an additional characteristic of the \gls{target} \gls{asic} is that the pedestal amplitude depends on the position in the waveform. The source of this characteristic is due to the fact that the storage cell blocks are not entirely decoupled from each other; the discharge of one block affects adjacent blocks. This effect is apparent in Figure~\ref{fig:cellwf}, where the pedestal amplitude of a single cell changes depending on the position of its parent block in the waveform. Consequently, an extra dimension of ``position in waveform'' must be considered in the waveform lookup table.

\subsubsection{Generation}

In order to perform the pedestal subtraction, one must first generate a lookup table of pedestal values. This can be easily obtained with a calibration run where the voltages across the photosensor are disabled, and forcing the camera to trigger (with either an external pulse generator, or internally via software) to obtain a large amount of waveform data. Typically around 30,000 events provide enough samples for every storage cell, in every waveform position, to have at least 10 entries. The samples are then collected as a running average with the dimensions $[Module, Channel, Starting Block, Blockphase+Sample\_i]$, where the $Starting Block$ is the storage block that the first sample in the waveform belongs to, $Blockphase$ is the cell index within the storage block that the waveform begins on, and $Sample\_i$ is the index of each sample in the waveform. This is illustrated in Figure~\change{include figure, and edit to use bp 8 and 12}, where for these two readout windows shown, the pedestal running average |Pedestal[TM][CHANNEL][9][8:103]| and |Pedestal[TM][CHANNEL][8][12:107]| will be contributed to, respectively.

The TargetCalib library handles the pedestal lookup table generation, and stores it into a \gls{fits} file. A new pedestal file is typically generated at the start of each new dataset, as the dependencies on temperature and evolution with time are still being investigated.

\subsubsection{Application}

To apply the pedestal, the entry within the lookup table that corresponds to each sample is subtracted from the waveform. The result of the subtraction can be seen in Figures~\ref{fig:pulse_raw_vs_pedestal}~and~\ref{fig:r1_cherenkov_image_mirrored_cropped}. 

\subsubsection{Performance}

\begin{figure}
	\centering
    \includegraphics[width=\textwidth]{pedestal_hist} 
	\caption[Spread of electronic-pedestal values before and after the pedestal subtraction.]{Spread of electronic-pedestal values before and after the pedestal subtraction for a single TARGET-C channel. The waveforms used to create the pedestal lookup table are from a different dataset to those used in these histograms. The mean $\mu$ and the standard deviation $\sigma$ of each distribution are also shown.} 
	\label{fig:pedestalresiduals}
\end{figure}

The primary quantification of this calibration's performance is the standard deviation of electronic-pedestal samples that have had separately-created pedestal values subtracted from them. Figure~\ref{fig:pedestalresiduals} demonstrates the performance of the pedestal subtraction for a \gls{targetc} channel, achieving a residual variation of \SI{1.59}{ADC} (approximately \SI{0.286}{\pe})\final{update value}.

\subsection{Transfer Function}

The other calibration related to the sampling and digitisation inside the \gls{target} \gls{asic} is caused by the non-linearities in the storing and reading of the analogue signal, to and from the storage cells (i.e. the charge and discharge of the switched capacitors). With reference to Figure~\ref{fig:target5diagram}, this means the non-linearity occurs in the steps between the sampling and storage array, and between the storage array the Wilkinson \glspl{adc}. The non-linearity of these components is propagated to the sample readout - a sample with twice the amplitude input into \gls{target} will have less than twice the amplitude when readout.

To correct for this non-linearity, a look-up table is generated to convert from the sample amplitude that is read out from the \gls{asic} (in \si{ADC}) to the sample amplitude that is input into the \gls{asic} (in \si{mV}). This look-up table is known as the Transfer Function. As one might expect, each sampling cell has its own linear response to account for, and therefore a look-up table is typically required at least per channel and per sampling cell, however a noticeably improved performance is observed by considering a Transfer Function per storage cell \change{update statement with actual result} \otherch{need to show this, maybe in TF Investigations appendix?}.

There are two forms of Transfer Function that have been considered for \gls{chec}, distinguished by the type of input used to generate them. A \gls{dc} Transfer Function is created by applying a constant \gls{dc} input of known voltage into the module, and iterating over the full dynamic range by varying the voltage. An \gls{ac} Transfer Function is generated by inputting a pulse of a known amplitude with a shape expected from the photosensor, and iterating as with the \gls{dc} approach. During previous investigations of the \gls{target} module, where sinusoidal signals were input into the module, a dependence on the signal frequency and input amplitude was observed that acts to further reduce the output amplitude \cite{Bechtol2012,Albert2017}. The source of this dependence was deemed to be due to the amplifiers, which cannot slew fast enough to keep up with the input signal if the frequency and amplitude are large. Due to the use of a pulse to generate the \gls{ac} Transfer Functions, the result inherently includes the correction required for the frequency that the pulses correspond to. 

\begin{figure}
  \begin{subfigure}[b]{0.49\textwidth}
    \includegraphics[width=\textwidth]{generation_t5}
    \caption{DC Transfer Function input, measured with TARGET-5.}
    \label{fig:generation_t5}
  \end{subfigure}
  \hfill
  \begin{subfigure}[b]{0.49\textwidth}
    \includegraphics[width=\textwidth]{generation_tc}
    \caption{AC Transfer Function input, measured with TARGET-C.}
    \label{fig:generation_tc}
  \end{subfigure}
  \caption[Transfer Function generation waveforms.]{Multiple average waveforms, increasing in amplitude. Each average contains 1000 waveforms from the same single channel. These waveforms cover the full dynamic range of the TARGET ASIC, and are used as inputs to generate the DC and AC Transfer Functions, respectively. The saturation behaviour of the TARGET-C ASIC can be seen in the high amplitude waveforms in (b).}
\end{figure}

\begin{figure}
  \begin{subfigure}[b]{0.49\textwidth}
    \includegraphics[width=\textwidth]{lookup_t5}
    \caption{DC Transfer Function lookup table, measured with TARGET-5. Contains 64 Transfer Functions, one for each Sampling Cell.}
    \label{fig:lookup_t5}
  \end{subfigure}
  \hfill
  \begin{subfigure}[b]{0.49\textwidth}
    \includegraphics[width=\textwidth]{lookup_tc}
    \caption{AC Transfer Function lookup table, measured with TARGET-C. Contains 4,096 Transfer Functions, one for each Storage Cell.}
    \label{fig:lookup_tc}
  \end{subfigure}
  \caption[Transfer Function lookup tables.]{The Transfer Function lookup tables for a single channel.}
\end{figure}

\subsubsection{Generation (\gls{dc} Transfer Function)}

During the commissioning of \gls{chec-m}, a \gls{dc} Transfer Function was used with no \gls{ac} corrections. To generate this Transfer Function, the internal input pedestal voltage (\gls{vped}) setting is used to apply a \gls{dc} voltage offset to the sampling \gls{asic}. This pedestal voltage is provided by a commercially obtained \gls{dac}, installed on the \gls{target5} module. This \gls{dac} has been characterised by the supplier, so the voltage amplitude obtained for each setting is known.

By repeating the process for Vped values from \SI{500}{mV} to \SI{1700}{mV}, in steps of \SI{25}{mV}, the full dynamic range of the module is explored, covering the range \SI{-250}{ADC} to \SI{3700}{ADC} (Figure~\ref{fig:generation_t5}). The running averages of the ADC samples are grouped and monitored according to $[Module, Channel, Sampling~Cell, Input~Amplitude]$, utilising every sample in the waveform. Around 1,000 events are required to provide sufficient statistics.

The second step in the generation of the \gls{dc} Transfer Function is to linearly interpolate the running averages at the ADC points defined by the user. This provides a lookup table of \si{mV} values with dimensions $[Module, Channel, Sampling~Cell, ADC~Value]$ that can be used to provide a calibrated value for a measured ADC value. The lookup table for a single channel is illustrated in Figure~\ref{fig:lookup_t5}. This table is saved to a \gls{fits} file, ready for application. A fresh \gls{dc} Transfer Function lookup table was typically created once a day during the \gls{chec-m} commissioning.

\subsubsection{Generation (\gls{ac} Transfer Function)}

\begin{figure}
	\centering
    \includegraphics[width=\textwidth]{tf_pulse_fit} 
	\caption[Fit of the waveform in order to extract samples to generate the \gls{ac} Transfer Function.]{An example of the amplitude extraction used for generating the \gls{ac} Transfer Function. The waveform is fit with two Landau functions (red curve). The samples of the waveform that occur at the time of the minimum and maximum of the fit (red arrows) are used as the inputs to the \gls{ac} Transfer Function.} 
	\label{fig:tf_pulse_fit}
\end{figure}

When the upgrade from the \gls{target5} module to \gls{targetc} was made, the commercially provided \gls{dac} for the setting of \gls{vped} was removed, and instead \gls{t5tea} generates a \gls{vped} internally itself. Contrary to the commercial \gls{dac}, the \gls{vped} provided by \gls{t5tea} is uncalibrated. Furthermore, the voltage applied is individual per channel, complicating the procedure to calibrate in. As a result, the approach of using the internal \gls{vped} setting to generate a \gls{dc} Transfer Function was abandoned. Instead, the decision was made to transition to an \gls{ac} Transfer Function that uses the expected pulse shape as an input. This approach therefore corrects for the \gls{ac} effect with the appropriate frequency. However, in order to externally input pulses from a pulse generator the module must be removed from the camera. Therefore, the \gls{ac} Transfer Function is only generated once in the present calibration pipeline.
	
The full dynamic range is once again probed, by injecting pulses of varying amplitude. In order to extract the values that correspond to negative amplitudes in this method, the amplitude of the input undershoot is also monitored. Only the samples that correspond to the maximum of the input pulse (and minimum of the undershoot) has a ``true'' amplitude of the input amplitude. Therefore, to extract the correct samples, each waveform is fitted with two Landau functions, a fair approximation to the pulse shape (Figure~\ref{fig:tf_pulse_fit}). Consequently, only two samples are extracted per waveform, requiring a much larger population of events (\utilde200,000) in order to generate a reliable running average grouped according to $[Module, Channel, Storage Cell, Input Amplitude]$. It is important to note that a Transfer Function per storage cell was adopted for \gls{targetc}, as it was found to significantly improve the residuals (see Appendix~\ref{a4-tf} for further discussion).

The second step in the generation of the \gls{ac} Transfer Function is identical to that in the \gls{dc} case. The resulting lookup table for a single channel can be seen in Figure~\ref{fig:lookup_tc}.

\subsubsection{Application}

Irrespective of the Transfer Function type, the lookup tables are stored in a format which enables them to be applied identically. When calibrating an ADC sample, the relevant lookup table is obtained according to the channel and cell of the sample, and is linearly interpolated to provide the calibrated \si{mV} value for the specified ADC value.

\subsubsection{Performance}

Due to its complexity and variety of approaches, the Transfer Function is still one of the most actively discussed aspects of the \gls{chec} calibration. Some possibilities for improvement include:
\begin{itemize}
	\item An improved sample extraction method for the \gls{ac} Transfer Function Waveform,
	\item The possibility for a \gls{dc} approach for \gls{targetc},
	\item Returning to the approach described in earlier \gls{target} studies where the pedestal is included inside the Transfer Function \cite{Albert2017},
	\item Alternatives to linear interpolation, such as Piecewise Cubic Hermite Interpolating Polynomial (PCHIP),
	\item Exchanging the lookup table for a parametrised regression characterisation of the Transfer Function (such as a high-order polynomial),
	\item Deciding between "per storage cell" or "per sampling cell",
	\item Inclusion of temperature corrections.
\end{itemize}

Assessing the performance of the Transfer Functions is a more complicated task than for the pedestals. We are no longer comparing to a null signal, and instead comparing to an input amplitude which contains its own uncertainty, and could potentially be incorrect. So while the performance results may indicate that the residuals of the Transfer Function are small, this does not necessarily mean the calibration is accurate. Therefore, the most decisive performance indicator should be one that provides an independent measurement on the ``correct'' amplitude. The most obvious scheme fitting this requirement is the \textit{Charge Resolution}, described in Chapter~\ref{ch3-architecture}, the results of which are explored in Appendix~\ref{a4-tf}.

\section{Photosensor Calibration} \label{section:photosensor_calib}

The other primary component in the detector chain that requires calibration is the photosensor itself. As photosensors are a much more common instrument used in a variety of experiments, the calibration procedures required are already well known in the academic community. It is therefore mostly a simple case of adapting existing approaches to fit our requirements.

The typical procedure in Cherenkov camera waveform analysis includes extracting the signal/charge from the waveform of each pixel. This procedure, and the different methods to achieve it, is described in Chapter~\ref{ch6-reduction}. The value extracted is typically in digitisation counts (ADC) or units of voltage, multiplied by time if the charge extraction approach is an integral over the waveform. For example, the units of the extracted charge from \gls{chec-s} using the \textit{Cross-Correlation} method (see Section~\ref{section:crosscorrelation}) is \si{mV ns}. Once extracted, this charge must be corrected for the relative efficiency of its pixel compared to the mean of the camera in order to achieve a uniform response (``flat-fielding''), and then converted into a counting unit that is common among the telescopes in the array (such as photons or photoelectrons), thereby simplifying the processing of array data \cite{Aharonian2004}. This procedure is characterised in the equation: 
\begin{equation} \label{eq:photosensor_calibration}
I_i = \frac{{A_Q}_i - {A_0}_i}{\gamma_{Q}} \times {\gamma_{FF}}_i,
\end{equation}
where 
\begin{itemize}
\item ${A_Q}_i$ is the charge extracted in units of \si{mV ns} for pixel $i$, proportional to the number of photoelectrons,
\item ${A_0}_i$ is the baseline in the absence of a signal for pixel $i$. It should be obtained using the same charge extraction approach used for the signal,
\item $\gamma_{Q}$ is the nominal conversion value from \si{mV ns} to photoelectrons/photons for the entire camera,
\item ${\gamma_{FF}}_i$ is the flat-field coefficient for the pixel $i$,
\item and $I_i$ is the resulting calibrated signal in photoelectrons/photons.
\end{itemize}

In the final calibration design of \gls{cta}, ${A_0}_i$ is intended to be supplied by the telescope alongside the waveforms at regular intervals. The regular updating of this value ensures that any changes to the baseline due to electronic noise, \gls{nsb} rate, or temperature variations (which can also increase \gls{dcr}, see Section~\ref{section:sipmt_parameters}) are accounted for. However, this parameter was set to zero for the content of this thesis, and was not investigated. Instead, a less effective but simpler baseline subtraction was performed by monitoring the running average of the first 16 samples of the past 50 waveforms for each pixel. This running average was subtracted from each waveform before charge extraction. The remainder of this section will describe how to obtain the other calibration values, $\gamma_{Q}$ and ${\gamma_{FF}}_i$, and the other procedures related to the photosensor calibration.

\subsection{Gain Matching}

\begin{figure}
	\centering
    \includegraphics[width=\textwidth]{before_after_gm} 
	\caption[Gain-Matching Residuals]{Comparison between the spread in the average signal amplitude per pixel before and after gain matching with \gls{chec-s}, for a dataset with approximately \SI{50}{\pe} average illumination. In the ``before'' case the \gls{dac} value in every superpixel was set to 100. Every pixel in the camera was included in the histogram. The mean $\mu$ and the standard deviation $\sigma$ of each distribution are also shown.} 
	\label{fig:before_after_gm}
\end{figure}

The flat-field coefficients, ${\gamma_{FF}}_i$, provide an offline compensation for the photosensor parameters which alter the signal response in the waveform. While this is typically only the gain in the case of \glspl{mapmt}, these parameters are more numerous for \glspl{sipmt}, and are described in Section~\ref{section:sipmt_parameters}. However, these parameters also have a dependence on the voltages across the photosensor, which is a controllable value. The dependence of the \gls{chec-s} \gls{sipmt} parameters on voltage is shown in Figure~\ref{fig:sipmt_checs}. With the \gls{chec-m} \glspl{mapmt} it is only possible to change the voltage value for an entire module, whereas with the \gls{chec-s} \glspl{sipmt} the voltages can be configured per superpixel (group of four pixels)\otherch{mention superpixels in ch3}. Therefore, voltage values can be selected before data-taking which result in a more uniform signal response between photosensor pixels. This is referred as ``Gain Matching'', however the name is slightly misleading, as it is the signal that is being matched, not the gain. It is performed by specifying the amplitude (in \si{mV}) that every pixel should be matched to, and then performing the following iterative procedure:
\begin{enumerate}
	\item The camera is uniformly illuminated with approximately \SI{50}{\pe}.
	\item The waveforms are readout, calibrated, and averaged per superpixel/module (excluding any dead pixels).
    \item The peak amplitudes of the average waveforms are extracted.
    \item Each module/superpixel is categorised as being above or below the requested amplitude.
    \item Depending on their category, the voltage setting is increased or reduced by steps of 5 (in arbitrary \gls{dac} units), such that it increments closer to the requested amplitude. If the amplitude has been overstepped in the previous measurement, a smaller step value is used. The minimum \gls{dac} step value available is 1, which corresponds to $\frac{10}{256}$~V. If the amplitude is not responding to changes in voltage, the pixel is classified as ``dead'', and excluded from the average waveforms.
    \item The new voltage settings are applied and the process is repeated.
\end{enumerate}

In the future, this iterative technique will be replaced with a set of lookup tables for different requested amplitudes. These lookup tables will contain the final voltage settings resulting from this iterative technique. Additionally in the future, the requested signal will not be specified in terms of peak amplitude, but in terms of the \textit{Cross Correlation} charge extraction approach. The resulting spread in signal response for \gls{chec-s} as a result of the gain matching is shown in Figure~\ref{fig:before_after_gm}.

The additional benefit of the gain matching is that it provides a convenient part in the data-taking chain to apply the bias compensation for temperature dependences (introduced in Section~\ref{section:sipmt_parameters}). This is achieved using the monitored temperature value per module (included in the data stream from the \gls{fee} modules) and a lookup table of the appropriate corrections to the voltages, such that a constant signal response is kept across the camera. This particular in-situ calibration has not yet been implemented, but is intended for the future.

\subsection{SPE Fitting}

\begin{figure}
	\centering
    \includegraphics[width=\textwidth]{spe_checm_checs} 
	\caption[Comparison of SPE spectra between CHEC-M and CHEC-S.]{Comparison of SPE spectra between CHEC-M and CHEC-S for a single pixel, along with their corresponding fit function.} 
	\label{fig:spe_checm_checs}
\end{figure}

Due to the photon-counting nature of \glspl{mapmt} and \glspl{sipmt}, when the signal extracted from a pixel, illuminated with a low light-level (\utilde\SI{1}{\pe}), is accumulated into a histogram, the resulting spectra (Figure~\ref{fig:spe_checm_checs}) show peaks at regular intervals corresponding to the baseline (zeroth peak), \SI{1}{\pe} (first peak), \SI{2}{\pe} (second peak), etc. As explained in Section~\ref{section:sipmt_parameters}, the single photoelectron resolution of \glspl{sipmt} is very high, much higher than is observed with \glspl{mapmt}. This accounts for the difference between the two photosensors in Figure~\ref{fig:spe_checm_checs}. These spectra are referred to as ``\gls{spe} Spectra''. The physical processes that result in these spectra are well understood for \glspl{mapmt} and \glspl{sipmt}, and therefore analytical formulae \final{check consistency in spelling} exist describing the spectra. When these formulae are fit to the histogram, they can be used to extract certain parameters of the photosensor, including the average incident illumination $\lambda$, in units of photoelectrons. As $\lambda$ provides an absolute illumination value, it allows for the full calibration of average expected charge for each filter-wheel position, for each pixel (conducted in Appendix~\ref{a2-lab}). This is the first step required in obtaining the flat-field coefficients. For more details on this fitting procedure, and the formulae used to describe the \gls{spe} spectra, refer to Appendix~\ref{a3-spe}.

\subsection{Flat-Field Coefficients}

\begin{figure}
	\centering
    \includegraphics[width=\textwidth]{flat_fielding} 
	\caption[Flat-field calibration]{The average measured charge per illumination for a single pixel. The Y error bars are the standard deviations of the charges for each illumination for a single pixel. The X error bars are the uncertainties on the average expected charge calibration (Section~\ref{section:fwerr}). The orange points were used in a linear regression through the origin to determine the flat-field coefficients for each pixel. The resulting gradient for the pixel ($\gamma_{M_i}$) is annotated.} 
	\label{fig:flat_fielding}
\end{figure}

\begin{figure}
	\centering
    \includegraphics[width=0.8\textwidth]{ff_values_cropped} 
	\caption[Flat-field Coefficients]{Camera image of the flat-field coefficient value, ${\gamma_{FF}}_i$, per pixel. Pixels that were designated ``dead'' or misbehaving are outlined in red, and exist beyond the colour-scale range.}
	\label{fig:ff_values}
\end{figure}

\begin{figure}
	\centering
    \includegraphics[width=\textwidth]{before_after_ff_50pe} 
	\caption[Flat-field residuals.]{Comparison between the spread in the average signal amplitude per pixel before (blue) and after (orange) the flat-fielding calibration. The charges were extracted from a dataset where a theoretical pixel located at the centre of the camera would be expected to have a charge of $Q_\text{Exp} \approx \SI{50}{\pe}$. The black histogram contains the charges after the difference in the illumination profile (Section~\ref{section:illumination_profile}) between the pixels was considered, i.e. they contain the charge that would be measured if every pixel was located at the camera centre. Every pixel in the camera, excluding the ``dead'' pixels, was included in the histograms. The mean $\mu$ and the standard deviation $\sigma$ of each distribution are also shown.}
	\label{fig:before_after_ff_50pe}
\end{figure}

\begin{figure}
  \begin{subfigure}[b]{0.49\textwidth}
    \includegraphics[width=\textwidth]{before_after_ff_25pe}
    \caption{Lower average expected charge}
    \label{fig:before_after_ff_25pe}
  \end{subfigure}
  \hfill
  \begin{subfigure}[b]{0.49\textwidth}
    \includegraphics[width=\textwidth]{before_after_ff_100pe}
    \caption{Higher average expected charge}
    \label{fig:before_after_ff_100pe}
  \end{subfigure}
  \caption[Flat-field residuals at other illuminations.]{Same as Figure~\ref{fig:before_after_ff_50pe}, but with a higher and lower average expected charge ($Q_\text{Exp}$).}
\end{figure}

Once the ``average expected charge'' dependence on filter-wheel position/transmission is characterised (Appendix~\ref{a2-lab}), we can calculate the coefficients, $\gamma_{M_i}$, required to convert the average measured charge (in \si{mV ns}) into the charge we expect (in photoelectrons/photons). The application of these coefficients to the extracted/measured charge has two effects:
\begin{itemize}
\item The signal response between pixels is homogenised - the same average amount of charge will be extracted for any pixel illuminated with an average of N photons.
\item The signal response is converted into the common telescope-array units of photoelectrons or photons.
\end{itemize}
Therefore:
\begin{equation} \label{eq:ff}
\gamma_{M_i} = \frac{\gamma_Q}{{\gamma_{FF}}_i}.
\end{equation}

To obtain $\gamma_{M_i}$ per pixel $i$ in the lab, datasets with around \SI{50}{\pe} average expected charge per pixel were produced. For each pixel, the average measured charge (in \si{mV ns}) was linearly regressed, while forcing the fit through the origin. This regression is shown for a single pixel in Figure~\ref{fig:flat_fielding}. The resulting gradient of the regression is equal to $\gamma_{M_i}$, which was combined with Equations~\ref{eq:photosensor_calibration}~and~\ref{eq:ff} for the calibration of measured charge into photoelectrons. The nominal conversion value from \si{mV ns} to photoelectrons for \gls{chec-s} was calculated to be $\gamma_Q = \SI[separate-uncertainty = true]{35.555 \pm 3.041}{mVns/\pe}$\final{update with latest}, and the spread of $\gamma_{FF_i}$ across the camera is shown in Figure~\ref{fig:ff_values}. The value for $\gamma_Q$ can be converted into its equivalent single \si{mV} sample (i.e. peak-height) equivalent using the reference pulse from the \textit{Cross Correlation} extraction method (Chapter~\ref{ch6-reduction}), resulting in a conversion value of \SI[separate-uncertainty = true]{4.373 \pm 0.374}{mV/\pe}\final{update with latest}.

The resulting residual spread in signal response between pixels at an average expected charge of \SI[separate-uncertainty = true]{47.67 \pm 3.79}{\pe} is shown in Figure~\ref{fig:before_after_ff_50pe}. The final variation in signal response between pixels at this illumination was measured to be \SI{0.5}{\percent}. Figures~\ref{fig:before_after_ff_25pe}~and~\ref{fig:before_after_ff_100pe} show the improvement of the average charge spread between pixels for a higher and a lower illumination.

As the flat-field coefficients have been calculated in a manner in which they are unfolded from the illumination profile (by calculating the average expected charge individually for each pixel), they are applicable to any environment the camera is used in. Any deviations that are measured in the signal between pixels are then due to the illumination profile present in the environment, and not due to the characteristics of the photosensor. Once the camera is on the telescope, the flat-field coefficients are intended to be routinely updated using the reflection of the LED flashers (Section~\ref{section:led_flashers}) in the secondary mirror. This calibration will require an updated illumination profile in order to be performed.

\subsubsection{Consideration of Errors and Uncertainty}

The standard error on the estimate of the gradient per pixel, $\sigma_{\gamma_{M_i}}$, that arises from a standard linear regression can be calculated with the relation derived by \textcite{Taylor1997}:
\begin{equation} \label{eq:merr}
\sigma_{\gamma_{M_i}} = \sigma_r \sqrt{\frac{N}{N \sum Q_{\text{Exp}_i}^2 - (\sum Q_{\text{Exp}_i})^2}}, \quad i = 0, 1, 2, ..., N,
\end{equation}
\begin{equation} \label{eq:sigmar}
\sigma_r = \sqrt{\frac{\sum (A_{Q_i} - A_{Q_f})^2}{N - 1}},
\end{equation}
where $N$ is the total number of regressed points $i$, $\sigma_r$ is the mean square error of the regression, the dependant variable $A_{Q_i}$ is the average measured charge at the average expected charge $Q_{\text{Exp}_i}$, and $A_{Q_f}$ is the value that results from the regression at that same value of $Q_\text{Exp}$. The denominator in Equation~\ref{eq:sigmar} is $N-1$ as we constrained the regression through the origin, therefore there was only one free parameter.

The error on $\gamma_{M_i}$ is used as weights when calculating the average to obtain $\gamma_Q$. Therefore the uncertainty on $\gamma_Q$ is quoted from the weighted standard deviation across the values of $\gamma_{M_i}$ for each pixel.

\subsection{Dead Pixels}

Figure~\ref{fig:ff_values} shows that some of the photosensor pixels contained either no signal or an odd signal, resulting in an extreme flat-field coefficient. This was likely due to damage to the pixel during handling, or due to water ingress. However, the four pixels constitute to \SI{0.2}{\percent} of the camera, therefore the camera is still well within the \requirementref{B-TEL-1295 Pixel Availability} \gls{cta} requirement (Section~\ref{section:pixel_availability}). These pixels were excluded from any calculations involving multiple pixels, including the expected-charge calibration and the charge-resolution across the camera. 

\section{Saturation Recovery} \label{section:saturation}

As evident in Figure~\ref{fig:flat_fielding}, high illumination measurements (greater than \utilde\SI{200}{\pe}) are affected by saturation of the detector. The saturation shown is due to the \gls{target} \gls{asic}, which saturates before the photosensor. However, while the height of the pulse increased no further, the excess charge caused the pulse to extend further (Figure~\ref{fig:generation_tc}). Therefore, it could be possible to perform a simple correction for the saturation recovery by utilising this waveform behaviour. A simple, initial investigation into saturation recovery is shown in Figure~\ref{fig:saturation_recovery}, where the waveform was integrated in a window that started just before the pulse maximum, and extended to the end of the waveform. This resulted in an extracted charge that continued to increase with illumination, apart from in the region immediately after saturation. More investigation is required for this calibration.

\begin{figure}[H]
	\centering
    \includegraphics[width=\textwidth]{saturation_recovery} 
	\caption[Saturation Recovery.]{Initial investigation into recovering charge from a saturated waveform for the same pixel as shown in Figure~\ref{fig:flat_fielding}. The saturation coefficient is the integral from just before the pulse maximum, to the end of the waveform readout.}
	\label{fig:saturation_recovery}
\end{figure}

\section{Timing Corrections} \label{section:timing_corrections}

\begin{figure}
	\centering
    \includegraphics[width=0.8\textwidth]{time_correction} 
	\caption[Pulse timing correction for each pixel.]{A camera image of \gls{chec-s} showing the timing correction for each pixel. The ``dead'' pixels are outlined in red, and have a zero timing correction.}
	\label{fig:time_correction}
\end{figure}

Due to the routing of the electronics in the front-end, the electrical signal path is slightly different per channel, causing a small difference in apparent arrival of the pulse in the waveform. The relative arrival time per pixel for \gls{chec-s} is shown in Figure~\ref{fig:time_correction}. This is measured by extracting the average arrival time per pixel over 1000 events at an average illumination of \SI{\sim 100}{\pe}, and subtracting each pixels value by the average across the camera. It is clear that in every module, there is one particular \gls{asic} slot, corresponding to a 16-pixel corner of the module, that has a longer electrical signal path. 

Not only does the timing correction need to be taken into consideration when investigating the timing performance, it also can have a significant impact on the charge extraction performance. This is because the charge extraction approaches typically rely on other pixels (neighbouring or entire camera, see Chapter~\ref{ch6-reduction}) sharing a compatible pulse time. A charge extraction routine that incorrectly extracts the charge by \SI{1}{ns} can have a negative impact on the \textit{Charge Resolution}. Discussions are ongoing on how to best include the timing corrections in the charge extraction.

\section{Future}

During the long development of \gls{chec}, the calibration procedure has evolved significantly. Multiple iterations of the procedures have occurred to:
\begin{itemize}
	\item Accommodate the changes required in the upgrades of hardware (such as from \gls{target5} to \gls{targetc}).
	\item Simplify the calibration to save on computing resources.
	\item Account for additional factors, thereby improving the calibration (such as the \gls{ac} contribution to the Transfer Functions).
\end{itemize}
Therefore, while each iteration improves in one aspect, it may be at the expense of the others. As a result, the \gls{target} calibration procedure described in this chapter appears quite complicated compared to the approaches detailed by \textcite{Bechtol2012} and \textcite{Albert2017}. The next step in the calibration development for \gls{chec} is therefore to review the procedure used, with the aim of producing an approach that is simpler, includes aspects such as temperature dependence, and meets the requirements and processing rates required by \gls{cta}.

%  \chapter{\label{ch6-reduction}Waveform Reduction} 

\minitoc

\section{Introduction}

Methods for retrieving information about the Cherenkov shower have been a primary component of the Imaging Atmospheric Cherenkov Technique since its inception. Early techniques such as those used in the first observation of TeV Gamma rays from the Crab nebula \cite{Weekes1989} are still utilised in modern \glspl{iact}. These techniques are also applicable to \gls{cta} telescopes, and as \gls{cta} is a large consortium which consists of the worldwide \gls{iact} community, the developers of the reduction approaches for previous \glspl{iact} have brought them forward to \gls{cta}. However, due to:
\begin{enumerate}[label=(\alph*)]
	\item \gls{cta} is to consist of the most advanced \glspl{iact} to date, with higher shower imaging resolution and telescope multiplicity than has previously been available,
	\item the capabilities of digital signal processing has significantly increased in the past decade,
\end{enumerate}
the opportunity for more advanced and more successful algorithms exists for \gls{cta}. Some effort has already been made in this direction, but it is an aspect that is expected to constantly evolve and improve during the lifetime of \gls{cta}. In this chapter I will provide insight into the existing and in-development reduction techniques utilised to extract the Cherenkov shower signal from the waveforms, and my involvement in designing a charge-extraction technique for \gls{chec-s}. I will also provide a brief overview of the shower reconstruction methods typically used with \glspl{iact}. With regards to the \gls{cta} data levels (Figure~\ref{fig:dataflow}), this chapter is mostly concerned with the steps from \textit{DL0} to \textit{DL2}.

\section{Charge Extraction Methods}

The immediate step after the waveform calibration is the extraction of signal from the waveforms provided by each camera individually. As a result of the low-level calibration detailed in \ref{ch5-calibration}, the waveforms from each \gls{cta} camera exist in a common state, with no remaining dependencies on the electronics they were produced with. Therefore, the extraction techniques are typically applicable to all \gls{cta} cameras. 

The extraction of signal from a waveform is a very generic problem, allowing for the utilisation of common signal processing techniques that are not unique to Cherenkov shower analysis. The goal is to extract as much signal from the pulse created by the Cherenkov shower light, while simultaneously limiting the inclusion of noise factors\change{reference where noise talked about, talk about all the contributing factors (noise etc.) to waveforms in ch2}. Two quantities are extracted in this stage: the signal charge in each pixel, and the signal arrival time per pixel. 

The total signal charge in a pixel,\change{Check all i.e. and e.g. for spaces} i.e.\@ the total number of photo-electrons released from the \gls{pmt}'s photocathode, is proportional to the total area below the pulse corresponding to the Cherenkov photons. If the waveforms were completely free of noise, and the readout window was large enough to capture the full Cherenkov signal, a simple integration of the entire readout would be a satisfactory approach for obtaining the signal charge. However, as we do not have the luxury of perfect waveforms, more complex methods are designed. Charge extraction algorithms typically consist of two aspects: how the signal pulse is found, and how the pulse is integrated.

\subsection{Peak Finding} \label{peakfinding}

Two factors must be considered when finding the signal pulse of a Cherenkov shower. Firstly, the majority of camera pixels will not contain any Cherenkov signal while still containing noise. A peak finding technique that assumes a signal exists in the readout will be biased, as it will mistake the noise for signal. Secondly, due to the nature of Cherenkov showers (Chapter~ \change{reference where the time gradient of Cherenkov showers are described}), those pixels with Cherenkov signal will have different arrival times due to the time evolution of the Cherenkov image \change{figure of peak time?}. This time gradient across the image is especially apparent for bright showers at a large core distance from the telescope. The most successful peak-finding technique is one that best accounts for those two factors. Some simple techniques used to define a peak time from a waveform include:
\begin{itemize}
	\item \textbf{Local Peak Finding}: Each waveform is treated independently from the other. The maximum point in the waveform is treated as the peak/arrival time. This approach is intrinsically biased to assume every waveform contains a signal; therefore, in the absence of a Cherenkov signal, the largest noise pulse will be extracted, resulting in a higher total charge than should be obtained.
	\item \textbf{Global Peak Finding}: The waveform from every pixel is combined into an average, from which the maximum point is treated as the peak time for every pixel. This technique is only useful if a large portion of the camera is simultaneously illuminated, such as by a laser in the case of lab commissioning and calibration runs.
	\item \textbf{Neighbour Peak Finding}: The waveforms from the neighbouring pixels are combined into an average, from which the maximum point is treated as the peak time for the pixel-of-interest. This technique is often preferred for Cherenkov images as it has a reduced charge bias (especially if the pixel-of-interest's waveform is not included in the average); pixels with Cherenkov signal typically have neighbours that also contain Cherenkov signal at a correlated time, while the neighbours of empty pixels only contain noise, and therefore a peak time that is uncorrelated to the noise is chosen.
	\item \textbf{Fixed Peak Value}: Due to a reliable definition of the camera trigger and subsequent electronic chain, the position of the pulse in the waveform could consistently be known a-priori, allowing for a fixed peak time. However, this method requires a larger integration window size in order to capture the full pulse in the tail of the Cherenkov shower, which occur at a later time than the initial photons which trigger the camera, therefore resulting in a larger noise included in the signal. However, this technique usually contains the least bias, as no signal is assumed to exist.
\end{itemize}

 A more complex peak-finding technique is the \textit{Gradient Peak Finding} approach. This approach was designed for the VERITAS telescope \cite{Holder2005}\cite{Cogan2006}\cite{Cogan2007}, but is applicable to any \gls{iact} telescope that allows the dynamic specification of an integration window. \textit{Gradient Peak Finding} utilises the gradient profile of the photon arrival time for gamma-induced Cherenkov showers, described in Section~\ref{section:photon_arrival_time} and illustrated in Figure~\change{chec-m pulse time figure}. This peak-finding technique is a two-pass approach performed by first extracting the signal using one of the other methods. The timing information contained within the pixels that survive cleaning (Section~\ref{section:image_cleaning}) can then be used to obtain a relation between ``distance along primary image axis'', $D_{ax}$, and the pulse time, $T_0$. Figure~\change{figure with geometry} illustrates the geometry of $D_{ax}$ with respect to the pixel position. Using the obtained relation between $D_{ax}$ and $T_0$, an example of which is shown in Figure~\change{figure of Dax versus T0 for checm figure}, an unbiased pulse time is obtained for each pixel depending on its position along the image axis.

These peak-finding methods have been described in relation to the maximum of the signal pulse, however they may instead use other characteristic positions of the pulse, such as the half-maximum time on the rising edge, or the centre of gravity of the pulse. Additionally, more advanced peak-finding techniques may up-sample (possibly by zero-padding in the frequency domain via a Fourier transform) or interpolate the signal to obtain a more precise peak time \cite{Cogan2006, Cogan2007}, or even apply low-pass filters in order to remove low frequency baseline noise. The peak finding should be done in conjunction with any timing corrections (\ref{section:timing_corrections}) that may be required.

\subsection{Integration}

Once the peak time has been obtained, the simplest approach to extract the signal is to define an integration window centred about this time. The size of the window needs to be large enough to capture sufficient signal from the pulse, but small enough that not too much noise (\gls{nsb}, dark counts, afterpulsing) is included within the window, thereby maximising the signal-to-noise. Additionally, the camera's pulse shape may not be symmetric, so a better signal-to-noise may be achieved by shifting the window a few samples with respect to the peak time. The optimal integration window size and shift for the \gls{chec-s} waveforms is found to be \SI{5}{samples} and \SI{2}{samples} (i.e.\@ \SI{5}{ns} and \SI{2}{ns}), respectively, according to the investigations performed in \ref{a3-extractors}.

Beyond the simple ``boxcar'' integrator method (where every sample integrated has a weight of 1), other more advanced strategies may define their own alternative approach to extract the charge. One example is the fitting of the signal pulse, with an analytical description of the expected pulse, or with a more unconstrained description such as a cubic spline. A second complex approach is the use of digital filters, which can be used in combination with knowledge of the pulse shape to robustly extract the signal even in the presence of high noise. Such a technique has been designed and adopted for \gls{gct}, referred to as the \textit{Cross-Correlation} method. Due to its adoption and sophistication, it is described here in more detail. 

\subsubsection{Cross Correlation} \label{section:crosscorrelation}

Cross-correlation is a common signal processing technique used as a measure of the similarity between two signals as a function of the displacement in time applied to one of the signals. Given a continuous function $f(t)$ defined between $0 \le t \le T$ and a second continuous function $g(t)$, the cross-correlation between the two functions ($f \star g$) is defined as 
\begin{equation} \label{eq:cc1}
(f \star g)(\tau) = \int_0^T \overline{f}(t)g(t + \tau)dt,
\end{equation}
where $\overline{f}(t)$ is the complex conjugate of $f(t)$ and $\tau$ is the time displacement (also referred to as the ``lag'') between the two functions \cite{wolfram-crosscorrelate}. In descriptive terms, by varying $\tau$, $g(t + \tau)$ will slide past $f(t)$. The cross-correlation for a value of $\tau_1$ is then the integral across $t$ of the product between $f(t)$ and $g(t + \tau_1)$. For a discreet function that is real-valued, such as a sampled waveform, Equation~\ref{eq:cc1} can instead be defined as
\begin{equation} \label{eq:cc2}
(f \star g)\lbrack n \rbrack = \sum_{m=0}^N f\lbrack m \rbrack g\lbrack m + n\rbrack,
\end{equation}
where $N$ is the total number of samples in the waveform and $m$ is the sample displacement. 

An illustration of the \textit{cross-correlation} approach being applied on a \gls{chec-s} waveform is shown in Figure~\change{figure showing the cross-correlation at a few different times, and the different stages, 3 axes, maybe for a low illumination? mention I implemented \ref{eq:cc2} in Python}. Through utilising a template of the expected pulse shape in the absence of noise (hereafter referred to as the ``reference pulse'') features inside the waveform that are correlated with the reference pulse shape are emphasized, while features that do not, such as the electronic noise, are suppressed. Therefore, the peaks in the cross-correlation result correspond to the displacements where the signals match best, and the values of the peaks correspond to an weighted integral of the entire waveform, and can be used as an extracted charge value.

The reference pulse we use for the cross-correlation was obtained via probing the input analogue signal on the \gls{target} module and averaged on an oscilloscope. It was then normalised such that cross-correlation between it, and the reference pulse normalised to have an integral of 1, has a maximum value of 1. This normalisation ensures that the cross-correlation result is in units of \si{mV ns}, and allows an easy conversion into \si{mV} for ``peak-height'' investigations.  An optimised implementation of cross-correlation exists in |scipy.ndimage.correlate1d| \cite{scipy-crosscorrelate}, where the waveforms for every pixel are processed in parallel.

\change{mention negatives of the cc approach, like the emphasis of nsb and cc, here or in appendix?}

\subsection{Approaches Adopted by Other IACTs}

Some examples of approaches adopted by other telescopes are outlined below, to provide an overview of techniques considered by other \glspl{iact}.

\subsubsection{MAGIC}

Members of the MAGIC telescope, \textcite{Albert2008}, performed a study comparing the techniques proposed for their signal reconstruction. Four approaches were compared: \textit{fixed-window}, \textit{sliding-window} with amplitude-weighted time, \textit{cubic spline fit} with integral or amplitude extraction, and \textit{digital filter}. It was concluded that the digital filter, which relies on knowledge of the signal shape to minimise the noise contributions, provided a charge reconstruction with acceptable bias and minimal variance, while remaining stable in the occurrence of small variations in pulse shape and position.

\subsubsection{VERITAS}

Similar to the aforementioned study for the MAGIC telescope, a comparison of charge extraction approaches was performed for VERITAS \cite{Holder2005, Cogan2006, Cogan2007}. Specifically, the extraction methods compared included a \textit{simple-window} using a-priori knowledge of the Cherenkov pulse time in the trace, a \textit{dynamic-window} which slides across the trace to find the Cherenkov pulse, a \textit{trace-fit} evaluator with fits the trace with two exponential functions which respectively describe the rise and fall time of the pulse, a \textit{matched-filter} which ``uses a digital filter based on the assumed shape of the FADC pulse to integrate the charge'' \cite{Cogan2007}, and finally an implementation of the \textit{Gradient Peak Finding} approach described earlier in the chapter. At first glance, some of these approaches bear resemblance to those used by MAGIC, however there are slight differences: 
\begin{itemize}
	\item in the VERITAS pulse fitting technique, an attempt to describe the pulse analytically was made whereas the MAGIC approach used a more loosely defined spline
	\item the filter used by VERITAS is a cross-correlation in Fourier space, whereas the filter used by MAGIC is generated using their knowledge of the noise auto-correlation matrix
\end{itemize}

Either as a result of these differences, or due to the difference in the instruments themselves, the VERITAS \textit{matched-filter} appears to result in a worse reconstruction than one would expect from the conclusion reached by MAGIC. One might justify that this degradation of signal extraction with the \textit{matched-filter} for higher amplitudes is due to a change in pulse shape at higher amplitudes, thereby requiring a different "assumed FADC pulse shape", but it is not clear if that is what is occurring here\change{Garret: However, a thorough investigation of this particular reconstruction is beyond the scope of this work}. These studies conclude that the \textit{matched-filter} ''holds promise`` for reconstructing low charges, whereas while the \textit{trace-fit} performs extremely poor for the low charges (as expected), it performs the best for amplitudes > 4 photoelectrons \cite{Cogan2007}.

\subsubsection{H.E.S.S.}

The standard mode of charge extraction for the \gls{hess} telescopes is to integrate $N$ samples with respect to a fixed, but regularly verified, signal time \cite{Aharonian2004}. \gls{hess} camera electronics underwent an upgrade in 2015/2016, subsequently allowing for the update of the standard extraction mode to also output time-of-maximum and time-over-threshold, and also allowed for full sample readout enabling the utilisation of more complex charge extraction techniques \cite{Klepser2017}\cite{Chalme-Calvet2015}.

\subsubsection{ASTRI}

Contrary to the other techniques described in this section, \gls{astri} took the alternative direction of a hardware-implemented charge extraction, utilising their \gls{citiroc} \gls{asic}. The pulse from their \gls{sipmt} is amplified and shaped (with both a high-gain and low-gain channel) with a constant shaping time of \SI{37.5}{ns}. The maximum of the shaped peak-height is then converted into an integrated charge, achieving no more than \SI{1}{\percent} introduced systematic error \cite{Impiombato2017}. The \textit{DL0} and \textit{DL1} formats are therefore identical in \gls{astri}'s case as the charge-extracted value is provided in place of waveforms. However, while this charge-extraction technique is optimised for the \gls{astri} electronics, it removes the flexibility of being able to dynamically select a charge-extraction technique within the software, and adopting new software-based techniques that may be designed in the future.

\subsection{Performance Assessment}

Deciding on which charge extraction method to use is not trivial - as shown in the above discussion, different cameras may perform better with different algorithms. This is anticipated in \pkg{ctapipe} (\ref{ch4-software}), where different |ChargeExtractors| can easily be selected at runtime depending on the camera source.

The assessment technique typically used for charge extractors in the context of \gls{cta} is the Charge Resolution (Section~\ref{section:cr}). A performance assessment of charge extraction techniques for \gls{chec-s} can be found in Appendix~\ref{a3-extractors}.

\section{Shower Reconstruction}

\begin{figure}
	\centering
    \includegraphics[width=\textwidth]{hillas} 
	\caption[Hillas Parametrisation Schematic.]{Schematic of the \textit{Hillas Parameters} used to describe the Cherenkov shower image ellipse.}
	\label{fig:hillas}
\end{figure}

\begin{figure}
\begin{minipage}[t]{.49\textwidth}
  \centering
  \includegraphics[width=0.92\textwidth]{source_location} 
  \captionof{figure}[Shower source location reconstruction.]{Reconstruction of the source location on the sky for the shower, achieved by superimposing the different views of the showers from different telescopes.}
  \label{fig:source_location}
\end{minipage}%
\hfill
\begin{minipage}[t]{.49\textwidth}
  \centering
  \includegraphics[width=\textwidth]{core_location}
  \captionof{figure}[Shower core location reconstruction.]{Reconstruction of the location of the shower core on the ground, achieved by combining the primary image axes at the relative telescope separations.}
  \label{fig:core_location}
\end{minipage}
\end{figure}

The resulting images of the extracted signal per triggered telescope is only the first stage of many to retrieve the properties of the Cherenkov-shower progenitor particle: direction, energy, and class. The direction is necessary to retrieve in order to obtain the source's position and spatial morphology. The energy is desired for studies of the source's spectrum. The class of the progenitor particle (gamma-ray, electron, or cosmic-ray hadron) is required in order to identify the gamma-rays among the hadronic background. All the information required to obtain these progenitor particle properties is contained within the image of extracted charge. As the focus of this thesis is on the low-level performance of the \gls{gct} cameras, this section will only include a brief overview of the simplest methods used in shower reconstruction. This information is supplied as context for the results described in \ref{ch8-pipeline}.

\subsection{Image Parametrisation}

Pixels containing signal from the Cherenkov shower are identified by an image cleaning method such as the tailcuts approach: any pixels above a certain signal threshold are kept, and any neighbouring pixels that are above a lower signal threshold are also kept. The thresholds are optimised per telescope using Monte Carlo simulations. The resulting pixels are then parametrised in terms of their second moments. This parametrisation is a predominant \gls{iact} analysis technique that has been utilised in the majority of \gls{iact} experiments. It was first formalised by \textcite{Hillas1985a} and has subsequently been known as the \textit{Hillas Parametrisation}. This technique exploits the elliptical shape of the gamma-induced shower images, and provides values for the centre of gravity of the ellipse, its primary axis position and orientation, its width, and its length (Figures~\ref{fig:hillas}). The \pkg{ctapipe} implementation of the \textit{Hillas Parametrisation} is defined by \textcite{Reynolds1993}.

\subsection{Direction Reconstruction}

Through utilising the primary image axis that results from the \textit{Hillas Parametrisation}, the positional information about the shower can be obtained \cite{Daum1997,Cogan2006,Dickinson2010}. By superimposing the ellipses from different telescopes onto a single image (Figure~\ref{source_location}), the stereoscopic combination allows for the source position of the shower to be retrieved with simple geometry. This is used to retrieve a gamma-ray's astrophysical-source position on the sky \change{perhaps refer to where it is explained that the two are the same, from introduction...}. If, instead, the primary image axes are combined with the relative positional view of the telescopes (Figure~\ref{fig:core_location}, the core location of the shower in terms of ground coordinates may be obtained. For obtaining the intersections of the axes in both applications, the weighted-mean-direction of the intersections is typically calculated \cite{Eschbach2016,Bernlohr2013a}. While the extraction of the directional information is much more reliable through the use of the stereoscopic combination of the images, it is not impossible to estimate the direction of the source when only a single telescope is available. The primary challenge when performing this operation with a single telescope is the degeneracy in determining which direction along the primary image axis the source exists in. This is overcome with the \textit{disp} method developed by \textcite{Lessard2001}.

\subsection{Energy Reconstruction}

As aptly described by \textcite[][p.~16]{Dickinson2010}, ``determining of the energy of the progenitor of a particular air shower relies on a form of electromagnetic calorimetry''. Once the distance from the telescopes to the shower is known (from the core position), it can be coupled with the \textit{image size} (the total amount of integrated signal obtained from the Cherenkov shower) to obtain total energy released as Cherenkov light \cite{Cogan2006,Bernlohr2013a}. This is a direct measure of the energy of the shower progenitor particle (Section~\change{section in introduction where it is shown that the majority of energy of the gamma-ray is released as Cherenkov light}).

\subsection{\textgamma-Hadron Discrimination}

\begin{figure}
	\centering
    \includegraphics[width=\textwidth]{gamma_hadron_discrimination} 
	\caption[Discriminating between images of gamma-ray and hadron induced showers.]{Example of the differences in the distributions of \textit{Hillas Parameters} between simulated \textgamma-ray-induced showers (dark) and real hadronic-induced showers (light) for the Whipple \SI{10}{m} reflector \cite{Fegan1999a}.}
	\label{fig:gamma_hadron_discrimination}
\end{figure}

The differences in image morphology between gamma-ray and hadronic showers is described in \ref{ch1-intro}\change{describe differences in images in intro, and show perfect examples with MC of CHECS. Lessard gives precise information about the differences}. These differences are quantified and encapsulated by the \textit{Hillas Parameters}, where gamma-ray showers exist within a narrow range of the parameters, while hadronic showers typically exhibit a broader spectrum of parameter values, as illustrated in Figure~\ref{fig:gamma_hadron_discrimination}. By creating an acceptance range for the \textit{Hillas Parameters}, the hadronic background can be excluded in a very simple way \cite{Dickinson2010,Fegan1999a,Hillas1996a}.

%  \chapter{\label{ch7-performance}Camera Performance} 

\minitoc

\notes[inline,caption={}]{
	\section{Plan}
	\subsection{Topics}
	\begin{itemize}
		\item Charge Resolution
		\item TF Investigations
		\item Different NSB
		\item MC Validation
		\item MC Performance
	\end{itemize}
	\subsection{Questions}
	\begin{itemize}
		\item What other criteria?
		\begin{itemize}
			\item Trigger performance - even though I haven't contributed
		\end{itemize}
	\end{itemize}
}

\section{Introduction}

\section{Pulse Shape}

\section{Timing Characteristics}

\section{MC Validation}

\change[inline]{big diff between MC and the lab is the potential cross talk in the electronics and the effect e.g. of ground bounce - difference in full camera illumination}


\section{Charge Resolution}

% The standard low-level criterion for performance used in \gls{cta} is the \textit{Charge Resolution}. It encompasses both the bias and the standard deviation of the extracted charge versus the expected charge to provide a measure of the waveform, calibration, and charge reconstruction quality. The \gls{cta} requirement \change[inline]{requirement label} (Chapter~\ref{ch3-architecture}) defines a limit which must be met for resolving the signal for any telescope in \gls{cta}.

\change[inline]{Justification of the lab chargeres using equation with ENF}

\notes[inline]{Remember to talk about how the MC charge resolution provides us insight into performance with perfect Transfer Functions}
\notes[inline]{Comparison in performance between CHEC-M and CHEC-S - show SPE spectrum (again), huge decrease in spe_sigma}

\section{Conclusion}
%  \chapter{\label{ch8-pipeline}On-Sky Pipeline} 

\minitoc

\notes[inline,caption={}]{
	\section{Plan}
	\subsection{Topics}
	\begin{itemize}
		\item Decided upon reduction methods
		\item Potentially different than for performance chapter
		\item CHEC-M campaign
		\item MC CHEC-S
		\item Future observations
		\item Jupiter observations (Beyond cherenkov?)
	\end{itemize}
	\subsection{Questions}
	\begin{itemize}
		\item ?
	\end{itemize}
}

\section{Introduction}

Installing the camera onto the telescope structure is a very important part of the comissioning procedure, as it tests the integration procedure
%  \chapter{\label{ch9-summary}Summary} 

\minitoc


Observations of the highest energy phenomena are facilitated by the SSTs


%%%%% APPENDICES
% \startappendices
% \include{text/a1-spe}
% \include{text/a2-tf}
% \chapter{\label{a3-extractors}Charge Extractor Investigations}

\minitoc

\notes[inline,caption={}]{
	\section{Plan}
	\subsection{Topics}
	\begin{itemize}
		\item RMS extracted charge with varying integration window size and shift with different NSBs and amplitudes (and combination of amplitudes)
		\item Charge resolution of different methods with different NSB (including full integration method)
		\item in the absence of the need for peak finding (laser illumination), and when peak finding is important (Cherenkov) - split into two, best integration technique (different NSBs, different integration window sizes), and best peak finding (different NSBs, need Cherenkov data)
		\item window size might actually change for Cherenkov - got to capture entire signal, and signal might not be centred at "calculated peak time".
	\end{itemize}
	\subsection{Questions}
	\begin{itemize}
		\item ?
	\end{itemize}
}

\section{Introduction}

In Chapter~\ref{ch6-reduction}, the different algorithms for extracting charge from a waveform is extensively discussed, as well as the important considerations one must be vigilant of. This Appendix is provided to inform about the performance of the chosen charge extraction approaches, \textit{Cross Correlation} and \textit{Neighbour Peak Finding}, in comparison to typical charge extraction approaches, in the context of \gls{chec-s}.

\section{Integration Window}

\begin{figure}
  \begin{subfigure}[b]{0.49\textwidth}
    \includegraphics[width=\textwidth]{generation_t5}
    \caption{DC Transfer Function input, measured with TARGET~5.}
    \label{fig:generation_t5}
  \end{subfigure}
  \hfill
  \begin{subfigure}[b]{0.49\textwidth}
    \includegraphics[width=\textwidth]{generation_tc}
    \caption{AC Transfer Function input, measured with TARGET~C.}
    \label{fig:generation_tc}
  \end{subfigure}
  \caption[Transfer Function generation waveforms.]{Multiple average waveforms, increasing in amplitude. Each average contains 1000 waveforms from the same single channel. These waveforms cover the full dynamic range of the TARGET ASIC, and are used as inputs to generate the DC and AC Transfer Functions, respectively. The saturation behaviour of the TARGET C ASIC can be seen in the high amplitude waveforms in (b).}
\end{figure}

As an alternative to the \textit{Cross Correlation} integration approach, one may instead use a simple integration window, defined by its ``window width'' and the ``shift'' from the peak time. The first step in this investigation is to find the optimal values for the width and shift of the window when extracting charge from a \gls{chec-s} waveform. This exercise is performed using a Monte Carlo simulation of the lab set-up, where the camera is uniformly illuminated. As the pulse time is consistent in this simulation, it allows the investigation to be solely on the integration approach, avoiding the contributions of peak finding.











%%%%% REFERENCES
\setlength{\baselineskip}{0pt}
{\renewcommand*\MakeUppercase[1]{#1}
\printbibliography[heading=bibintoc,title={\bibtitle}]}


\end{document}
