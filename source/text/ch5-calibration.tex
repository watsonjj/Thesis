\chapter{\label{ch5-calibration}Calibration} 

\minitoc

\notes[inline,caption={}]{
	\section{Plan}
	\subsection{Topics}
	\begin{itemize}
		\item Pedestal subtraction
		\item Transfer functions
		\item Gain Matching
		\item SPE
		\item Flat fielding
		\item Time correction
		\item Future
		\begin{itemize}
			\item Live calibration
		\end{itemize}
	\end{itemize}
	\subsection{Questions}
	\begin{itemize}
		\item TARGET architecture diagram, Wilkinson ADC 
		\item How much detail about all the TF approaches do I go into?
	\end{itemize}
}

\section{Introduction}

In order to obtain meaningful and reliable results from the camera, a number of calibrations must be applied to the waveforms read. A primary objective of my DPhil is to investigate the most optimal and efficient approaches for these calibrations (in accordance with the \gls{cta} requirements described in Chapter~\ref{ch3-architecture}), and to determine if additional calibrations are required.

The calibrations applied have evolved during the course of the prototyping of \glsr{chec}; the calibrations applied to \gls{chec-m} waveforms are not the same for \glsr{chec-s}. Additionally, the calibration applied for the on-sky pipeline can differ slightly to the calibration used to obtain results such as the charge resolution.

In this chapter I will outline the each of the calibration steps in the general order that they are applied.


\section{TARGET Calibration}

The calibrations described in this section relate to the \glsr{target} module. As detailed in Chapter~\ref{ch2-mechanics}, the \glsr{target} \gls{asic} is responsible for the sampling and digitisation of the waveforms, and therefore 

\subsection{Electronic Pedestal Subtraction}

The most important, but also the simplest calibration to apply is the subtraction of the electronic pedestal